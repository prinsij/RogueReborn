	\documentclass[12pt, titlepage]{article}

% Packages

\usepackage{booktabs}
\usepackage{tabularx}
\usepackage{hyperref}
\usepackage{indentfirst}
\usepackage[round]{natbib}

% Colour Scheme

\hypersetup{
    colorlinks,
    citecolor=black,
    filecolor=black,
    linkcolor=red,
    urlcolor=blue
}

% Custom Commands

\newcounter{NFRCounter}
\newcounter{FRCounter}

\newcolumntype{R}[1]{>{\raggedleft\let\newline\\\arraybackslash\hspace{0em}}p{#1}}
\newcolumntype{L}[1]{>{\raggedright\let\newline\\\arraybackslash\hspace{0em}}p{#1}}

\newcommand{\freq}[1]{
	\hfill\stepcounter{FRCounter}FR.\textbf{\theFRCounter}
	& #1
}

\newcommand{\spec}[3]{
	\stepcounter{NFRCounter}
	\begin{center}
		\def\arraystretch{1.6}
		\begin{tabular}{| R{7em} L{20em} |}
			\bottomrule
			\multicolumn{2}{| c |}{\textbf{Non-Functional Requirement} \# \theNFRCounter}  \\
			\hline
			\textit{Description:} & #1 \\
			\textit{Rationale:} & #2 \\
			\textit{Fit Criterion:} & #3 \\
			\toprule
		\end{tabular}
	\end{center}
}

% Document Details

\title{SE 3XA3: Requirements Specification\\Rogue Reborn}

\author{Group \#6, Team Rogue++\\\\
	\begin{tabular} {l r}
		Ian Prins & prinsij \\
		Mikhail Andrenkov & andrem5 \\
		Or Almog  & almogo
	\end{tabular}
}

\date{Due Tuesday, October 11\textsuperscript{th}, 2016}

%% Comments

\usepackage{color}

\newif\ifcomments\commentstrue

\ifcomments
\newcommand{\authornote}[3]{\textcolor{#1}{[#3 ---#2]}}
\newcommand{\todo}[1]{\textcolor{red}{[TODO: #1]}}
\else
\newcommand{\authornote}[3]{}
\newcommand{\todo}[1]{}
\fi

\newcommand{\wss}[1]{\authornote{blue}{SS}{#1}}
\newcommand{\ds}[1]{\authornote{red}{DS}{#1}}
\newcommand{\mj}[1]{\authornote{red}{MSN}{#1}}
\newcommand{\cm}[1]{\authornote{red}{CM}{#1}}
\newcommand{\mh}[1]{\authornote{red}{MH}{#1}}

% team members should be added for each team, like the following
% all comments left by the TAs or the instructor should be addressed
% by a corresponding comment from the Team

\newcommand{\tm}[1]{\authornote{magenta}{Team}{#1}}


\begin{document}

\maketitle

\pagenumbering{roman}
\tableofcontents
\listoftables
\listoffigures

\begin{table}[bp]
	\caption{\bf Revision History}
	\bigskip
	\begin{tabularx}{\textwidth}{p{3cm}p{2cm}X}
		\toprule {\bf Date} & {\bf Version} & {\bf Notes}\\
		\midrule
			09/28/16 & 1.0 & Initial Setup\\
			10/02/16 & 1.0 & Continued Setup\\
			10/07/16 & 1.1 & Added Project Drivers\\
			10/07/16 & 1.1 & Added Functional Requirements and Risks\\
			10/09/16 & 1.2 & Added Non-Functional Requirements\\
			10/10/16 & 1.3 & Added 4.1-4.5 \\
			10/11/16 & 1.4 & Added 4.9,4.10,2.1.* \\
		\bottomrule
	\end{tabularx}
\end{table}

\newpage

\pagenumbering{arabic}

This document describes the requirements for the Rogue Reborn project.  The template for the Software Requirements Specification (SRS) is a subset of the Volere template~\citep{RobertsonAndRobertson2012}.  For the convenience of the readers, the sections pertaining to the non-functional requirements have been expanded into their respective subsections with respect to the Volere template.

\section{Project Drivers}

	\subsection{The Purpose of the Project}

	The goal of the project is to produce a reimplementation of the original Rogue computer game, originally developed by Michael Toy, Glenn Wichman, and Ken Arnold in 1980. The gameplay of the reimplementation should mimic that of the original whenever possible. The objective of the rewrite is to produce a copy in a modern language, using modern design principles, with superior documentation and a full test suite. The original Rogue is of historical interest as it forms the foundation and is the namesake of the roguelike genre of games, typified by their randomized environments, difficulty, and permadeath features. The motivation for this project is the poor condition of the original source code. The original source was not written with readability in mind, and designed for extremely low-performance systems who required some unusual design patterns. The version of C in which it was written is very old, which hinders compilation or feature extension. The intended audience for this document is the stakeholders of this project, especially Dr Smith and the 3XA3 TAs.

	\subsection{The Stakeholders}

		\subsubsection{The Client}

		The client of the project is Dr Spencer Smith. Dr Smith commissioned the project and will be overseeing its production. Dr Smith provides the specifications for this document, as well as other aspects of the project, including the test suite, and all documentation. In addition he will be evaluating the final product.

		\subsubsection{The Customers}

		The project customers are the players of the game. It is expected that this will consist primarily of players of the original, as well as players and developers of later roguelike games. The roguelike community has a strong open-source tradition, so a modern, well-documented Rogue could be a valuable starting point or inspiration for projects by other teams.

		\subsubsection{Other Stakeholders}

Other stakeholders include playtesters of the game, as well as the 3XA3 TAs. Playtesters of the game will be recruited to play the game, and therefore have stake in the success of the project. The 3XA3 TAs will be evaluating the success of the project, as well as providing feedback and guiding the project while it is still in development.

	\subsection{Mandated Constraints}

As a constraint imposed by the project client, there are a number of deadlines for the project throughout its development. In particular, the final demonstration of functionality will be on november the 30th, and the final draft of the project documentation must be produced by the 8th of december. The goal of replicating the gameplay of the original without significant change restricts the platforms for which the project can be developed. In particular, the interface for the original is extremely ill-suited to touch-input environments such as phones and tablets. 

	\subsection{Naming Conventions and Terminology}

	Listed below are a number of video game and/or roguelike specific terms used in this document. 
	\begin{itemize}
	\item libtcod: a.k.a. "The Doryen Library", libtcod is a popular and feature-rich library for roguelike development, with bindings for C, C++, Python, Lua, and C\#. 
	\item Rogue: Both the name of the 1980 computer game, and the a reference to the player character (we will always use the term player character).
	\item Roguelike: A genre of games similar to Rogue. Membership in the roguelike genre is largely determined by the presence or absence of permadeath, but many games feature many more similarities.
	\item Permadeath: A feature of roguelikes where the game is restarted from the beginning upon character death.
	\item Hitpoints: A positive integer value that measures the health of a character (more is healthier).
	\item Strength: A key statistic of the player character, strength determines how likely they are to successfully land a hit with a melee weapon and how much damage it is likely to do.
	\item Item Identification: A common feature of roguelikes where items are scrabbled at the beginning of a game, with the player not knowing which corresponding to which effect. Certain effects or simply using these items can identify items. For example, a blue potions may be potions of healing in one game, but in the next they could be sleeping gas. Item identification also refers to determining whether a given item is cursed.
	\item Cursed equipment: Equipment that once used reveal itself to be harmful to their user and difficult to remove.
	\item Dungeon: Consisting of a stack of 30 floors, the dungeon forms the game world in Rogue.
	\item Gold: Gold coins can be found throughout the dungeon, the number of gold coins collected is the primary basis for the player's score.
	\item Level: Can refer to a floor of the dungeon, or to the player character's experience level, which determines their hitpoints.
	\item Experience: Experience is gained by defeating monsters, and sufficient quantities will cause the player character to level up.
	\item Searching: Certain features of the dungeon, such as traps and hidden doors are not immediately visible to the player. The player character can explicitly search their immediate surroundings for such features.
	\end{itemize}

	\subsection{Relevant Facts and Assumptions}

It is assumed users will be utilizing the product in a 64 bit Linux environment, with a keyboard and monitor of at least 1280x400 pixels. Users are assumed to be at least moderately familiar with the original or similar games, as no extra material describing how to play the game is planned to be produced.

\section{Functional Requirements}

	\subsection{The Scope of the Work and the Product}

		\subsubsection{The Context of the Work}

		The context of the work has vastly changed since the original Rogue came out in 1980. Firstly, the times are very different now. Whereas in the 1980's computers were far and few to find, today they play an irreplaceable part of our society. People are on average a lot more familiar with computers than they were back then, therefore the possible market of users is significantly larger.\\

		On the topic of markets, the video game industry has grown tremendously into a international multi-billion dollar industry. The humble Rogue is faced with giants in the field, and while none capture the same magic as the original dungeon crawler, there are certainly other large players on the field.\\

		The final contextual aspect to consider is the thematic inspirations of Rogue. Rogue takes place in a realm of fantasy, drawn up primarily from some high-fantasy setting of Dungeons and Dragons, which itself has drawn much from various works, such as Tolkien's \textit{The Lord of the Rings}, \textit{The Hobbit}, and \textit{The Silmarillion}. Since the release of Rogue in 1980, many more modern pieces in the genera have been released, such as George R. R. Martin's \textit{A Song of Ice and Fire}, and the collective works of R.A. Salvatore. The influence of these new works can be found in extensions over the original Rogue, such as \textit{Moria} (1983).

		\subsubsection{Work Partitioning}

		The work required to complete this project has been divided up between the three group members Ian, Mikhail, and Or. Each has been assigned a highly-cohesive, loosely coupled segment of the code that is to be written. It was unanimously agreed that each team member is to present his API to the rest of the team as soon as time permits. This "API" materializes as a C++ header file, with which other modules in the code can interact.\\

		\begin{itemize}
			\item Or will be in charge of dungeon generations. This includes generating rooms, corridors, walls, doors, handling vision, and the placement of treasure and traps.
			\item Mikhail will be in charge of most player-tangibles. This includes eating, quaffing potions, handling weaponry, using armor, rings, wands, and scrolls. Much of this realm also crosses over to monster actions, which Mikhail will be in charge of as well.\\
			\item Ian will be in charge of the game's state control. The flow of the game, the timing of events, and highscores will all fall under his domain.\\
		\end{itemize}
		
		\subsubsection{Individual Product Use Cases}

		The product will have one primary use: playing the game. This is the most direct path to completion of the objective, which is to supply entertainment to the user. Most users, as may be anticipated, will do nothing with the project besides this. However, as experience always shows, alternative uses exist for everything. During the 1980's, a group of college students built a piece of software that had one goal: beat the original Rogue game. With the ever-growing advancements in artificial intelligence of today's modern world, it would not be completely foolish to suggest that an AI could potentially be built for this edition. In fact, one could argue that if a new AI system were to be designed to beat Rogue, its designers would seek out this new version, as it would supply a well-documented API with which the system could interact.

	\subsection{Functional Requirements}

	This section will specify the functional requirements of the Rogue++ project. They are numerous, scattered, and interdependent, therefore an attempt shall be made to organize them into cascading, logical segments.

		\subsubsection{Basic mechanics}
			\begin{tabular}{p{0.1\linewidth}p{0.9\linewidth}}
				\freq{The player should be able to start a new game}\\
				\freq{The player should be able to save the current game by name}\\
				\freq{The player should be able to load previous games by name}\\
				\freq{The player should be able to quit the game}\\
				\freq{The player must always begin with the default level 1 hero}\\
				\freq{The player must always see their hero's statistics}\\
				\freq{The game must wait until the user takes an action to manipulate the environment}\\
				\freq{The game must be able to present a help menu}\\
			\end{tabular}

		\subsubsection{Interaction}
			\begin{tabular}{p{0.1\linewidth}p{0.9\linewidth}}
				\freq{The player should be able to view detailed information about the hero}\\
				\freq{The player should be able to view detailed information about the surrounding environment}\\
				\freq{The player should be able to pass the turn}\\
				\freq{The player should be able to walk around}\\
				\freq{The player should be able to open and close doors}\\
				\freq{The player must be able to fall under status effects}\\
			\end{tabular}

		\subsubsection{The Dungeon}
			\begin{tabular}{p{0.1\linewidth}p{0.9\linewidth}}
				\freq{The player must begin at the dungeon's first level}\\
				\freq{The game must generate each dungeon level one at a time}\\
				\freq{Each level must have a downwards staircase}\\
				\freq{Every level must generate rooms, corridors, monsters, treasure, and traps}\\
				\freq{The player must be able to see in a 3x3 square centered on the hero}\\
				\freq{The player must be able to see the entire room the hero is in, if the hero is in a room}\\
				\freq{The player should see the \textit{outline} of dungeon areas previously explored}\\
				\freq{The player should be able to search for hidden doors and traps}\\
				\freq{The player should not be able to see hidden doors without explicitly searching for them}\\
				\freq{The Amulet of Yendor must be generated in level 26}\\
			\end{tabular}

		\subsubsection{Equipment}
			\begin{tabular}{p{0.1\linewidth}p{0.9\linewidth}}
				\freq{The game should maintain an inventory of player items}\\
				\freq{The player should be able to view the inventory}\\
				\freq{The game should limit the player's inventory based on the weight of its contents}\\
				\freq{The player should be able to add, drop, use, hold, and remove objects from the inventory}\\
				\freq{Scrolls, rings, and wands should have meaningless names until identified}\\
				\freq{Scrolls, rings, and wands should be usable}\\
				\freq{The player should be able to identify items}\\
				\freq{The player should not be able to remove cursed items}\\
				\freq{Player armor should be able to deteriorate}\\
			\end{tabular}

		\subsubsection{Combat}
			\begin{tabular}{p{0.1\linewidth}p{0.9\linewidth}}
				\freq{Each monster must have its own statistics}\\
				\freq{Each monster must calculate a plan of action}\\
				\freq{Monsters must only attack the player, not other monsters}\\
				\freq{Every in-game entity must be defeatable}\\
				\freq{The player must re-gain lost health over time}\\
				\freq{Armor must reduce the damage taken by the player}\\
			\end{tabular}

\newpage

\section{Non-functional Requirements}

	\subsection{Look and Feel Requirements}
		\subsubsection{Appearance Requirements}
			\spec{The Rogue Reborn UI shall closely resemble the original \textit{Rogue} UI.}{The new game should be visually similar to the old game.}{The new UI must have similar locations for all GUI elements and must use ASCII symbols for all graphical components.}

		\subsubsection{Style Requirements}
			There are no significant requirements that are applicable to this category.

	\subsection{Usability and Humanity Requirements}
		\subsubsection{Ease of Use Requirements}
			\spec{Rogue Reborn shall be fun and entertaining.}{Games are developed for enjoyment purposes.}{The game must be able to hold the interest of a new user for at least 20 minutes.}

		\subsubsection{Personalization and Internationalization Requirements}
			\spec{Rogue Reborn shall target an anglophone audience.}{The game will be developed and tested by an anglophone population.}{All game text must be written in English, free of any grammar or spelling mistakes.}

		\subsubsection{Learning Requirements}
			\spec{The Rogue Reborn game shall be easy to learn and play.}{Users may prematurely lose interest in the game if the controls are difficult or frustrating.}{The game must use an intuitive keyboard layout and possess an in-game mechanism to view all key bindings.}
			
		\subsubsection {Understandability and Politeness Requirements}
			There are no significant requirements that are applicable to this category.

		\subsubsection {Accessibility Requirements}
			There are no significant requirements that are applicable to this category.

	\subsection{Performance Requirements}
		\subsubsection{Speed and Latency Requirements}
			\spec{Rogue Reborn shall appear responsive to user input.}{Slow update times may induce frustration.}{On average, the game UI must be updated within at least 33ms of a visible user action.}
			
		\subsubsection{Safety-Critical Requirements}
			There are no significant requirements that are applicable to this category.

		\subsubsection{Precision or Accuracy Requirements}
			\spec{Rogue Reborn shall use integer types with an appropriate level of precision.}{Integer overflow may cause unexpected behaviour.}{All integer values in the game with an unknown upper bound must be at least 32 bits in size.}

		\subsubsection{Reliability and Availability Requirements}
			\spec{Rogue Reborn shall not crash under normal operating circumstances.}{Frequent crashes may frustrate users and diminish their experience.}{Every reproducible event that causes the game to crash must be documented, root-caused, and resolved.}

		\subsubsection{Robustness or Fault-Tolerance Requirements}
			There are no significant requirements that are applicable to this category.

		\subsubsection{Capacity Requirements}
			\spec{Rogue Reborn shall be able to record the high scores of up to 15 users.}{Allows for a variety of users to directly compete against one another.}{The game must be able to load and display the high scores of 15 previous performances.}

		\subsubsection{Scalability or Extensibility Requirements}
			There are no significant requirements that are applicable to this category.

		\subsubsection{Longevity Requirements}
			There are no significant requirements that are applicable to this category.

	\subsection{Operational and Environmental Requirements}

		\subsubsection{Expected Physical Environment}
			\spec{Rogue Reborn shall successfully run on any modern laptop or desktop computer with an Intel x64 processor.}{Most potential users will have access to this hardware environment.}{The game must display stable behaviour on a computer with an Intel x64 processor (equipped with a keyboard, mouse, and monitor).}

		\subsubsection{Requirements for Interfacing with Adjacent Systems}
			There are no significant requirements that are applicable to this category.

		\subsubsection{Productization Requirements}
			\spec{Rogue Reborn shall be distributed as a compressed folder containing a single executable file along with any necessary licenses.}{This is a simple approach to the distribution process.}{The game must be distributed as a folder containing a collection of applicable licenses in addition to a single executable file that is able to run on a fresh system without any external dependencies.}

		\subsubsection{Release Requirements}
			There are no significant requirements that are applicable to this category.

	\subsection{Maintainability and Support Requirements}

		\subsubsection{Maintenance Requirements}
			\spec{All reported bugs shall be resolved within a month of their submission.}{Immediately concentrating effort on subcritical bugs may distract developers.}{Every incident featured in the GitLab ITS must be closed within a month of its creation.}

		\subsubsection{Supportability Requirements}
			There are no significant requirements that are applicable to this category.

		\subsubsection{Adaptability Requirements}
			\spec{Rogue Reborn shall successfully run on a modern Linux x64 operating system.}{It is assumed that the product testers and consumers will have access to a Linux x64 operating system.}{The game must display stable behaviour on an Ubuntu x64 distribution.}

	\subsection{Security Requirements}
		\subsubsection{Access Requirements}
			There are no significant requirements that are applicable to this category.

		\subsubsection{Integrity Requirements}
			\spec{Rogue Reborn shall verify the validity of the saved high score file before displaying its contents.}{Malicious users may attempt to inject false records into this file.}{The game must display no previous high scores if it detects a flaw in the records file.}

		\subsubsection{Privacy Requirements}
			There are no significant requirements that are applicable to this category.

		\subsubsection{Audit Requirements}
			There are no significant requirements that are applicable to this category.

		\subsubsection{Immunity Requirements}
			There are no significant requirements that are applicable to this category.

	\subsection{Cultural Requirements}
		There are no significant requirements that are applicable to this category, since Rogue Reborn does not modify any cultural aspects from the original \textit{Rogue}.

	\subsection{Legal Requirements}
		\subsubsection{Compliance Requirements}
			\spec{Rogue Reborn shall be distributed with an accompanying \href{../../LICENSE.txt}{LICENSE.txt} file.}{This license must be distributed with projects that are a modification of the original \textit{Rogue} source code.}{The corresponding \href{../../LICENSE.txt}{LICENSE.txt} file is included in the distribution package.}

		\subsubsection{Standards Requirements}
			There are no significant requirements that are applicable to this category.

	\subsection{Health and Safety Requirements}
		\spec{Rogue Reborn shall not contain visual sequences that are likely to trigger seizures.}{Individuals with photosensitive epilepsy may feel disoriented, uncomfortable, or unwell ~\citep{PhotosensitiveEpilepsy}.}{The average luminosity of the game UI cannot change by more than 0.5 between two successive frames.}

\section{Project Issues}

	\subsection{Open Issues}

		The most pressing issue is whether the project will include a Windows version of the product. Issues linking with the libtcod library on Windows have put this item into doubt. Whether save compatability will be maintained with the original is also an open issue, this is likely to be an expensive feature in relation to the value it adds to the project. 

	\subsection{Off-the-Shelf Solutions}

		We have chosen to use the libtcod library for this project as an off-the-shelf solution to some problems in the product. Libtcod providers a high-level, cross-platform abstraction over rendering and user input, as well as a number of utilities such as line-drawing and pathfinding. There are a number of ports of Rogue to various platforms, including one that upgrades to graphics of the game to graphical tiles, but as of writing we are not aware of any code-cleanup focused rewrite.

	\subsection{New Problems}

		So long as the project requirements are met, especially the health and safety requirements, the product should not adversely affect the user. There may be issues with building and deploying the project, as of writing the product has not been tested without building from source. This could potentially require a partial rewrite of the project. It is also unlikely but possible that the product may corrupt the user's files in some way when attempting to save or load a game.

	\subsection{Tasks}

		As outlined by the project client, the project is split into a number of development phases. An early proof of concept will be produced first, followed by a test plan for the product, then final development and documentation. This proof of concept phase will consist largely on laying the foundation for the various systems in the product. For example, basic combat will be in the proof of concept, but more advanced combat such as thrown/ranged weapons and monster abilities will be left for later. The full development should consist largely of fleshing in these systems, developing tests, and more advanced features. Development tasks within a phase will be partitioned among team members as the team leader sees fit. For more detail on the proof of concept and other aspects of the development see the Development Plan.

	\subsection{Migration to the New Product}

		Migration to the new product should not be an issue for users of the original Rogue. While it is an open issue whether save files will be compatible across versions, since games of Rogue rarely last longer than a few hours this is unlikely to be a major issue for users. It is a goal of the project that the user interface of the product should be unchanged from the original, so migrating users should have minimal issues learning to use the product. Users not familiar with the original may find the product (particularly the user interface), somewhat confusing, but since Rogue was released over 30 years ago, there are a number of resources available online which explain the interface and the basic gameplay. It is the intention of this project for the product to be available in a format that can be simply compiled by any 64-bit Linux system with a C++ compiler, so the installation process should not be a burden.

	\subsection{Risks}

		\begin{itemize}
			\item \textbf{Computer Usage Risks} - There are several risks associated with computer usage. This is often a subject matter that is discussed thoroughly in an office environment, where computers see frequent, daily usage.
			\begin{itemize}
				\item When using a computer, there is an ergonomic risk involved. Improper usage of the computer can lead to aches in various parts of the body, including back, neck, hands, and chest.
				\item There is also a significant risk of eye aches, along with other vision problems.
				\item Repetitive motion is another factor that could cause discomfort when using a computer.
			\end{itemize}
			\item \textbf{Offensive Content} - The game draws heavily from fantasy, involving themes of violence, fear, and witchcraft. While these elements are only displayed in a textual context, certain cultures and societies may find such elements offensive or disturbing.
			\item \textbf{Anger} - The game is not easy. Frustration could easily overcome the player, especially when he/she has progressed far into the game. Anger management issues are widespread, and evidence of anger due to video games is easily found.
		\end{itemize}

	\subsection{Costs}

		The project's costs will be extremely limited. With the original being open source, there are no licensing concerns to worry about. In addition, all software used in the project is free and potentially open-source. The only potential costs involved is the electricity required to run the development machines.

	\subsection{User Documentation and Training}

		If a modern user tried to play the original Rogue, they would not have easy time getting started. The controls are not intuitive, and the interface even less so. Luckily for the user, the final product will include an in-game help menu, to help players with getting started. This menu's primary purpose will be to brief the user of the controls. After reviewing the menu, the user should be fairly capable of playing the game. With this basic training, the user should be able to discover all of the game's functionalities.

	\subsection{Waiting Room}

		The waiting room prescribes objectives, requirements, and features that could be implemented in future iterations. The following is a list of such features, for which consideration was given but time could not allow for:
		\begin{itemize}
			\item \textbf{More Monsters} - The original Rogue has 1 monster per letter of the alphabet, for a total of 26. This is very small number, especially when compared to modern video games. By using different colours or a tile-set (see 4.10), it is possible to achieve a far greater number of enemies to challenge the player.

			\item \textbf{Infinite Descent} -  The concept of a never-ending game is not new. And when dungeon levels are generated on the fly, would not be hard to implement at all. An infinite dungeon would turn the mastery of the game around. Currently, the ''best'' run-through is the one that accomplishes the end goal in the fewest moves and most gold. With infinite descent possible, the best run would knock out end goal from the equation - turning the goal of the game into a strict function of acquired wealth. As the player descends, the levels would become progressively more difficult, and generate more gold for the player to collect.

			\item \textbf{Seed Sharing} -  A seed is a sequence of bits that dictate which numbers the ''random'' number generator will create. If two separate pseudo-random number generators use the same seed for instantiation, the two will generate the same ''random'' numbers. A direct consequence of this is the ability to generate the same dungeon repeatedly, all one needs is the seed. This would be useful for determining who really is the better Rogue player, as some dungeons may require more steps to complete the end goal.\\

			As a side note, for this to work, dungeon generation must not rely on the state of the player, for surely different players will take on the dungeon in different ways.
		\end{itemize}

	\subsection{Ideas for Solutions}

		\begin{itemize}
			\item \textbf{Graphics} - Modern video games have engaging animations, special effects, life-like detail, and an overall immersive user experience. Rogue's peak graphical experience is an ASCII symbol surrounded by some capital letters. It is no question that graphics are a primary bottleneck for attracting users. For this reason, it would be wise to inquire about using a tile-set for graphics. This is something libtcod is capable of doing. Using a 16x16, 32x32, or even 64x64 tileset will vastly improve the graphical user experience.

			\item \textbf{Language Translations} - The project is presently written in English, but support for more languages is a reasonable feature to have. Having more language support would open up accessibility to more users and encourage engagement.

			\item \textbf{Tutorial Mode} - There is no denying it: Rogue is a difficult game. It is frustrating and hard to understand, yet rewarding at the end of it all. Overcoming the initial barrier to play is critical. Introducing a tutorial mode would be supremely beneficial to new players learning the ropes.
		\end{itemize}

\newpage

\

\bibliographystyle{plainnat}

\bibliography{SRS}

\newpage

\section{Appendix}

This section has been added to the Volere template.  This is where you can place
additional information.

	\subsection{Symbolic Parameters}

	The definition of the requirements will likely call for SYMBOLIC\_CONSTANTS.
	Their values are defined in this section for easy maintenance.


\end{document}
