\documentclass[12pt, titlepage]{article}

% Packages

\usepackage{booktabs}
\usepackage{tabularx}
\usepackage{hyperref}
\usepackage{indentfirst}
\usepackage[round]{natbib}

% Colour Scheme

\hypersetup{
    colorlinks,
    citecolor=black,
    filecolor=black,
    linkcolor=red,
    urlcolor=blue
}

% Custom Commands

\newcounter{NFRCounter}

\newcolumntype{R}[1]{>{\raggedleft\let\newline\\\arraybackslash\hspace{0em}}p{#1}}
\newcolumntype{L}[1]{>{\raggedright\let\newline\\\arraybackslash\hspace{0em}}p{#1}}

\newcommand{\spec}[3]{
	\stepcounter{NFRCounter}
	\begin{center}
		\def\arraystretch{1.6}
		\begin{tabular}{| R{7em} L{20em} |}
			\bottomrule
			\multicolumn{2}{| c |}{\textbf{Non-Functional Requirement} \# \theNFRCounter}  \\
			\hline
			\textit{Description:} & #1 \\
			\textit{Rationale:} & #2 \\
			\textit{Fit Criterion:} & #3 \\
			\toprule
		\end{tabular}
	\end{center}
}

% Document Details

\title{SE 3XA3: Requirements Specification\\Rogue Reborn}

\author{Group \#6, Team Rogue++\\\\
	\begin{tabular} {l r}
		Ian Prins & prinsij \\
		Mikhail Andrenkov & andrem5 \\
		Or Almog  & almogo
	\end{tabular}
}

\date{Due Tuesday, October 11\textsuperscript{th}, 2016}

\input{../Comments}

\begin{document}

\maketitle

\pagenumbering{roman}
\tableofcontents
\listoftables
\listoffigures

\begin{table}[bp]
	\caption{\bf Revision History}
	\bigskip
	\begin{tabularx}{\textwidth}{p{3cm}p{2cm}X}
		\toprule {\bf Date} & {\bf Version} & {\bf Notes}\\
		\midrule
			09/28/16 & 1.0 & Initial Setup\\
			10/02/16 & 1.0 & Continued Setup\\
			10/07/16 & 1.1 & Added Functional Requirements and Risks\\
			10/09/16 & 1.2 & Added Non-Functional Requirements\\
		\bottomrule
	\end{tabularx}
\end{table}

\newpage

\pagenumbering{arabic}

This document describes the requirements for the Rogue Reborn project.  The template for the Software Requirements Specification (SRS) is a subset of the Volere template~\citep{RobertsonAndRobertson2012}.  For the convenience of the readers, the sections pertaining to the non-functional requirements have been expanded into their respective subsections with respect to the Volere template.

\section{Project Drivers}

	\subsection{The Purpose of the Project}

	The goal of the project is to produce a reimplementation of the original Rogue computer game, originally developed by Michael Toy, Glenn Wichman, and Ken Arnold in 1980. The gameplay of the reimplementation should mimic that of the original whenever possible. The objective of the rewrite is to produce a copy in a modern language, using modern design principles, with superior documentation and a full test suite. The original Rogue is of historical interest as it forms the foundation and is the namesake of the roguelike genre of games, typified by their randomized environments, difficulty, and permadeath features. The motivation for this project is the poor condition of the original source code. The original source was not written with readability in mind, and designed for extremely low-performance systems who required some unusual design patterns. The version of C in which it was written is very old, which hinders compilation or feature extension. The intended audience for this document is the stakeholders of this project, especially Dr Smith and the 3XA3 TAs.

	\subsection{The Stakeholders}

		\subsubsection{The Client}

		The client of the project is Dr Spencer Smith. Dr Smith commissioned the project and will be overseeing its production. Dr Smith provides the specifications for this document, as well as other aspects of the project, including the test suite, and all documentation. In addition he will be evaluating the final product.

		\subsubsection{The Customers}

		The project customers are the players of the game. It is expected that this will consist primarily of players of the original, as well as players and developers of later roguelike games. The roguelike community has a strong open-source tradition, so a modern, well-documented Rogue could be a valuable starting point or inspiration for projects by other teams.

		\subsubsection{Other Stakeholders}

	\subsection{Mandated Constraints}

	\subsection{Naming Conventions and Terminology}

	\subsection{Relevant Facts and Assumptions}

	User characteristics should go under assumptions.

\section{Functional Requirements}

	\subsection{The Scope of the Work and the Product}

		Not sure about this section

		\subsubsection{The Context of the Work}

		\subsubsection{Work Partitioning}

		\subsubsection{Individual Product Use Cases}

	\subsection{Functional Requirements}

	This section will specify the functional requirements of the Rogue++ project. They are numerous, scattered, and interdependent, therefore an attempt shall be made to organize them into cascading, logical segments.

		\subsubsection{Basic mechanics}

		\begin{itemize}
			\item The player should be able to start a new game
			\item The player should be able to save the current game by name
			\item The player should be able to load previous games by name
			\item The player should be able to quit the game
			\item The player must always begin with the default level 1 hero
			\item The player must always see their hero's statistics
			\item The game must wait until the user takes an action to manipulate the environment
		\end{itemize}

		\subsubsection{Interaction}
		\begin{itemize}
			\item The player should be able to view detailed information about:
			\begin{itemize}
				\item The hero
				\item The surrounding environment
			\end{itemize}
			\item The player should be able to pass the turn
			\item The player should be able to walk around
			\item The player should be able to open and close doors
		\end{itemize}

		\subsubsection{The Dungeon}
			\begin{itemize}
				\item The player must begin at the dungeon's first level
				\item The game must generate each dungeon level one at a time
				\item Each level must have a downwards staircase
				\item Every level must generate rooms, corridors, monsters, treasure, and traps
				\item The player must be able to see in a 3x3 square centered on the hero
				\item The player must be able to see the entire room the hero is in, if the hero is in a room
				\item The player should see the outline of dungeon areas previously explored
				\item The player should be able to search for hidden doors
				\item The player should not be able to see hidden doors without explicitly searching for them
			\end{itemize}

		\subsubsection{Equipment}
			\begin{itemize}
				\item The game should maintain an inventory of player items
				\item The player should be able to view the inventory
				\item The game should limit the player's inventory based on the weight of its contents
				\item The player should be able to add, drop, use, hold, and remove objects from the inventory
				\item Scrolls, rings, and wands should have meaningless names until identified
				\item The player should be able to identify items
				\item The player should not be able to remove cursed items
			\end{itemize}

		\subsubsection{Combat}
		\begin{itemize}
			\item Each monster must have its own statistics
			\item Each monster must calculate a plan of action
			\item Monsters must only attack the player, not other monsters

		\end{itemize}

\newpage

\section{Non-functional Requirements}

	\subsection{Look and Feel Requirements}
		\subsubsection{Appearance Requirements}
			\spec{The Rogue Reborn UI shall closely resemble the original \textit{Rogue} UI.}{The new game should be visually similar to the old game.}{The new UI must have similar locations for all GUI elements and must use ASCII symbols for all graphical components.}

		\subsubsection{Style Requirements}
			There are no significant requirements that are applicable to this category.


	\subsection{Usability and Humanity Requirements}
		\subsubsection{Ease of Use Requirements}
			\spec{Rogue Reborn shall be fun and entertaining.}{Games are developed for enjoyment purposes.}{The game must be able to hold the interest of a new user for at least 20 minutes.}

		\subsubsection{Personalization and Internationalization Requirements}
			\spec{Rogue Reborn shall target an anglophone audience.}{The game will be developed and tested by an anglophone population.}{All game text must be written in English, free of any grammar or spelling mistakes.}

		\subsubsection{Learning Requirements}
			\spec{The Rogue Reborn game shall be easy to learn and play.}{Users may prematurely lose interest in the game if the controls are difficult or frustrating.}{The game must use an intuitive keyboard layout and possess an in-game mechanism to view all key bindings.}
			
		\subsubsection {Understandability and Politeness Requirements}
			There are no significant requirements that are applicable to this category.

		\subsubsection {Accessibility Requirements}
			There are no significant requirements that are applicable to this category.
		

	\subsection{Performance Requirements}
		\subsubsection{Speed and Latency Requirements}
			\spec{Rogue Reborn shall appear responsive to user input.}{Slow update times may induce frustration.}{On average, the game UI must be updated within at least 33ms of a visible user action.}
			
		\subsubsection{Safety-Critical Requirements}
			There are no significant requirements that are applicable to this category.

		\subsubsection{Precision or Accuracy Requirements}
			\spec{Rogue Reborn shall use integer types with an appropriate level of precision.}{Integer overflow may cause unexpected behaviour.}{All integer values in the game with an unknown upper bound must be at least 32 bits in size.}

		\subsubsection{Reliability and Availability Requirements}
			\spec{Rogue Reborn shall not crash under normal operating circumstances.}{Frequent crashes may frustrate users and diminish their experience.}{Every reproducible event that causes the game to crash must be documented, root-caused, and resolved.}

		\subsubsection{Robustness or Fault-Tolerance Requirements}
			There are no significant requirements that are applicable to this category.

		\subsubsection{Capacity Requirements}
			\spec{Rogue Reborn shall be able to record the high scores of up to 15 users.}{Allows for a variety of users to directly compete against one another.}{The game must be able to load and display the high scores of 15 previous performances.}

		\subsubsection{Scalability or Extensibility Requirements}
			There are no significant requirements that are applicable to this category.

		\subsubsection{Longevity Requirements}
			There are no significant requirements that are applicable to this category.

	\subsection{Operational and Environmental Requirements}

		\subsubsection{Expected Physical Environment}
			\spec{Rogue Reborn shall successfully run on any modern laptop or desktop computer with an Intel x64 processor.}{Most potential users will have access to this hardware environment.}{The game must display stable behaviour on a computer with an Intel x64 processor (equipped with a keyboard, mouse, and monitor).}

		\subsubsection{Requirements for Interfacing with Adjacent Systems}
			There are no significant requirements that are applicable to this category.

		\subsubsection{Productization Requirements}
			\spec{Rogue Reborn shall be distributed as a compressed folder containing a single executable file along with any necessary licenses.}{This is a simple approach to the distribution process.}{The game must be distributed as a folder containing a collection of applicable licenses in addition to a single executable file that is able to run on a fresh system without any external dependencies.}

		\subsubsection{Release Requirements}
			There are no significant requirements that are applicable to this category.

	\subsection{Maintainability and Support Requirements}

		\subsubsection{Maintenance Requirements}
			\spec{All reported bugs shall be resolved within a month of their submission.}{Immediately concentrating effort on subcritical bugs may distract developers.}{Every incident featured in the GitLab ITS must be closed within a month of its creation.}

		\subsubsection{Supportability Requirements}
			There are no significant requirements that are applicable to this category.

		\subsubsection{Adaptability Requirements}
			\spec{Rogue Reborn shall successfully run on a modern Linux x64 operating system.}{It is assumed that the product testers and consumers will have access to a Linux x64 operating system.}{The game must display stable behaviour on an Ubuntu x64 distribution.}


	\subsection{Security Requirements}
		\subsubsection{Access Requirements}
			There are no significant requirements that are applicable to this category.

		\subsubsection{Integrity Requirements}
			\spec{Rogue Reborn shall verify the validity of the saved high score file before displaying its contents.}{Malicious users may attempt to inject false records into this file.}{The game must display no previous high scores if it detects a flaw in the records file.}

		\subsubsection{Privacy Requirements}
			There are no significant requirements that are applicable to this category.

		\subsubsection{Audit Requirements}
			There are no significant requirements that are applicable to this category.

		\subsubsection{Immunity Requirements}
			There are no significant requirements that are applicable to this category.

	\subsection{Cultural Requirements}
		There are no significant requirements that are applicable to this category, since Rogue Reborn does not modify any cultural aspects from the original \textit{Rogue}.

	\subsection{Legal Requirements}
		\subsubsection{Compliance Requirements}
			\spec{Rogue Reborn shall be distributed with an accompanying \href{../../LICENSE.txt}{LICENSE.txt} file.}{This license must be distributed with projects that are a modification of the original \textit{Rogue} source code.}{The corresponding \href{../../LICENSE.txt}{LICENSE.txt} file is included in the distribution package.}

		\subsubsection{Standards Requirements}
			There are no significant requirements that are applicable to this category.


	\subsection{Health and Safety Requirements}
		\spec{Rogue Reborn shall not contain visual sequences that are likely to trigger seizures.}{Individuals with photosensitive epilepsy may feel disoriented, uncomfortable, or unwell ~\citep{PhotosensitiveEpilepsy}.}{The average luminosity of the game UI cannot change by more than 0.5 between two successive frames.}

\section{Project Issues}

	\subsection{Open Issues}

	\subsection{Off-the-Shelf Solutions}

	\subsection{New Problems}

	\subsection{Tasks}

	\subsection{Migration to the New Product}

	\subsection{Risks}

	\begin{itemize}
		\item \textbf{Computer Usage Risks} - There are several risks associated with computer usage. This is often a subject matter that is discussed thoroughly in an office environment, where computers see frequent, daily usage.
		\begin{itemize}
			\item When using a computer, there is an ergonomic risk involved. Improper usage of the computer can lead to aches in various parts of the body, including back, neck, hands, and chest.
			\item There is also a significant risk of eye aches, along with other vision problems.
			\item Repetitive motion is another factor that could cause discomfort when using a computer.
		\end{itemize}
		\item \textbf{Offensive Content} - The game draws heavily from fantasy, involving themes of violence, fear, and witchcraft. While these elements are only displayed in a textual context, certain cultures and societies may find such elements offensive or disturbing.
		\item \textbf{Anger} - The game is not easy. Frustration could easily overcome the player, especially when he/she has progressed far into the game. Anger management issues are widespread, and evidence of anger due to video games is easily found.
	\end{itemize}

	\subsection{Costs}

	

	\subsection{User Documentation and Training}

	\subsection{Waiting Room}

	\subsection{Ideas for Solutions}

\newpage

\bibliographystyle{plainnat}

\bibliography{SRS}

\newpage

\section{Appendix}

This section has been added to the Volere template.  This is where you can place
additional information.

	\subsection{Symbolic Parameters}

	The definition of the requirements will likely call for SYMBOLIC\_CONSTANTS.
	Their values are defined in this section for easy maintenance.


\end{document}