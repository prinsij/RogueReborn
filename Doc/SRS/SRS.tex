\documentclass[12pt, titlepage]{article}

\usepackage{booktabs}
\usepackage{tabularx}
\usepackage{hyperref}
\hypersetup{
    colorlinks,
    citecolor=black,
    filecolor=black,
    linkcolor=red,
    urlcolor=blue
}
\usepackage[round]{natbib}

\title{SE 3XA3: Requirements Specification\\Rogue Reborn}

\author{Group \#6, Team Rogue++\\\\
	\begin{tabular} {l r}
		Ian Prins & prinsij \\
		Mikhail Andrenkov & andrem5 \\
		Or Almog  & almogo
	\end{tabular}
}

\date{Due Friday, October 7\textsuperscript{th}, 2016}

%% Comments

\usepackage{color}

\newif\ifcomments\commentstrue

\ifcomments
\newcommand{\authornote}[3]{\textcolor{#1}{[#3 ---#2]}}
\newcommand{\todo}[1]{\textcolor{red}{[TODO: #1]}}
\else
\newcommand{\authornote}[3]{}
\newcommand{\todo}[1]{}
\fi

\newcommand{\wss}[1]{\authornote{blue}{SS}{#1}}
\newcommand{\ds}[1]{\authornote{red}{DS}{#1}}
\newcommand{\mj}[1]{\authornote{red}{MSN}{#1}}
\newcommand{\cm}[1]{\authornote{red}{CM}{#1}}
\newcommand{\mh}[1]{\authornote{red}{MH}{#1}}

% team members should be added for each team, like the following
% all comments left by the TAs or the instructor should be addressed
% by a corresponding comment from the Team

\newcommand{\tm}[1]{\authornote{magenta}{Team}{#1}}


\begin{document}

\maketitle

\pagenumbering{roman}
\tableofcontents
\listoftables
\listoffigures

\begin{table}[bp]
	\caption{\bf Revision History}
	\bigskip
	\begin{tabularx}{\textwidth}{p{3cm}p{2cm}X}
		\toprule {\bf Date} & {\bf Version} & {\bf Notes}\\
		\midrule
			09/28/16 & 1.0 & initial setup\\
			10/02/16 & 1.0 & Continued setup\\
			10/07/16 & 1.0 & Fleshing out section 1\\
		\bottomrule
	\end{tabularx}
\end{table}

\newpage

\pagenumbering{arabic}

This document describes the requirements for the Rogue++ project  The template for the Software
Requirements Specification (SRS) is a subset of the Volere
template~\citep{RobertsonAndRobertson2012}. No further modifications have been made from the blank
project template given.

\section{Project Drivers}

\subsection{The Purpose of the Project}

The goal of the project is to produce a reimplementation of the orignal Rogue computer game, originally developed by Michael Toy, Glenn Wichman, and Ken Arnold in 1980. The gameplay of the reimplementation should mimic that of the original whenever possible. The objective of the rewrite is to produce a copy in a modern language, using modern design principles, with superior documentation and a full test suite. The original Rogue is of historical interest as it forms the foundation and is the namesake of the roguelike genre of games, typified by their randomized environments, difficulty, and permadeath features. The motivation for this project is the poor condition of the original source code. The original source was not written with readability in mind, and designed for extremely low-performance systems who required some unusual design patterns. The version of C in which it was written is very old, which hinders compilation or feature extension. The intended audience for this document is the stakeholders of this project, especially Dr Smith and the 3XA3 TAs.

\subsection{The Stakeholders}

\subsubsection{The Client}

The client of the project is Dr Spencer Smith. Dr Smith commissioned the project and will be overseeing its production. Dr Smith provides the specifications for this document, as well as other aspects of the project, including the test suite, and all documenation. In addition he will be evaluating the final product.

\subsubsection{The Customers}

The project customers are the players of the game. It is expected that this will consist primarily of players of the original, as well as players and developers of later roguelike games. The roguelike community has a strong open-source tradition, so a modern, well-documented Rogue could be a valuable starting point or inspiration for projects by other teams.

\subsubsection{Other Stakeholders}

Other stakeholders include playtesters of the game, as well as the 3XA3 TAs. Playtesters of the game will be recruited to play the game, and therefore have stake in the success of the project. The 3XA3 TAs will be evaluating the success of the project, as well as providing feedback and guiding the project while it is still in development.

\subsection{Mandated Constraints}

As a constraint imposed by the project client, there are a number of deadlines for the project throughout its development. In particular, the final demonstration of functionality will be on november the 30th, and the final draft of the project documentation must be produced by the 8th of december. The goal of replicating the gameplay of the original without significant change restricts the platforms for which the project can be developed. In particular, the interface for the original is extremely ill-suited to touch-input environments such as phones and tablets. 

\subsection{Naming Conventions and Terminology}

\subsection{Relevant Facts and Assumptions}

It is assumed users will be utilizing the product in a 64 bit Linux environment, with a keyboard and monitor of at least [INSERT DIMENSIONS]. Users are assumed to be at least moderately familiar with the original, no extra material describing how to play the game is planned to be produced.

User characteristics should go under assumptions. [DELETE]

\section{Functional Requirements}

\subsection{The Scope of the Work and the Product}

\subsubsection{The Context of the Work}

\subsubsection{Work Partitioning}

\subsubsection{Individual Product Use Cases}

\subsection{Functional Requirements}

\section{Non-functional Requirements}

\subsection{Look and Feel Requirements}

\subsection{Usability and Humanity Requirements}

\subsection{Performance Requirements}

\subsection{Operational and Environmental Requirements}

\subsection{Maintainability and Support Requirements}

\subsection{Security Requirements}

\subsection{Cultural Requirements}

\subsection{Legal Requirements}

\subsection{Health and Safety Requirements}

This section is not in the original Volere template, but health and safety are
issues that should be considered for every engineering project.

\section{Project Issues}

\subsection{Open Issues}

\subsection{Off-the-Shelf Solutions}

\subsection{New Problems}

\subsection{Tasks}

\subsection{Migration to the New Product}

\subsection{Risks}

\subsection{Costs}

\subsection{User Documentation and Training}

\subsection{Waiting Room}

\subsection{Ideas for Solutions}

\newpage

\bibliographystyle{plainnat}

\bibliography{SRS}

\newpage

\section{Appendix}

This section has been added to the Volere template.  This is where you can place
additional information.

\subsection{Symbolic Parameters}

The definition of the requirements will likely call for SYMBOLIC\_CONSTANTS.
Their values are defined in this section for easy maintenance.


\end{document}