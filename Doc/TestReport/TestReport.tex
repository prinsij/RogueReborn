\documentclass[12pt, titlepage]{article}

\usepackage{booktabs}
\usepackage{tabularx}
\usepackage{float}
\usepackage{hyperref}
\usepackage{indentfirst}
\usepackage{listings}
\usepackage[round]{natbib}
\usepackage[usenames, dvipsnames]{color}
\usepackage{tikz}
\usepackage{longtable}

% Setup

\hypersetup{
    colorlinks,
    citecolor=blue,
    filecolor=ForestGreen,
    linkcolor=MidnightBlue,
    urlcolor=blue
}

\lstset{
	basicstyle=\ttfamily\footnotesize
}

\usepackage[round]{natbib}

\title{SE 3XA3: Test Plan\\Rogue Reborn}

\author{Group \#6, Team Rogue++\\\\
	\begin{tabular} {l r}
		Ian Prins & prinsij \\
		Mikhail Andrenkov & andrem5 \\
		Or Almog & almogo
	\end{tabular}
}

\date{Due Wednesday, Dec 7\textsuperscript{st}, 2016}

%% Comments

\usepackage{color}

\newif\ifcomments\commentstrue

\ifcomments
\newcommand{\authornote}[3]{\textcolor{#1}{[#3 ---#2]}}
\newcommand{\todo}[1]{\textcolor{red}{[TODO: #1]}}
\else
\newcommand{\authornote}[3]{}
\newcommand{\todo}[1]{}
\fi

\newcommand{\wss}[1]{\authornote{blue}{SS}{#1}}
\newcommand{\ds}[1]{\authornote{red}{DS}{#1}}
\newcommand{\mj}[1]{\authornote{red}{MSN}{#1}}
\newcommand{\cm}[1]{\authornote{red}{CM}{#1}}
\newcommand{\mh}[1]{\authornote{red}{MH}{#1}}

% team members should be added for each team, like the following
% all comments left by the TAs or the instructor should be addressed
% by a corresponding comment from the Team

\newcommand{\tm}[1]{\authornote{magenta}{Team}{#1}}


\begin{document}

\maketitle

\pagenumbering{roman}
\tableofcontents
\listoftables
\listoffigures

\begin{table}[bp]
	\caption{\bf Revision History}
	\begin{tabularx}{\textwidth}{p{3cm}p{2cm}X}
		\toprule {\bf Date} & {\bf Version} & {\bf Notes}\\
		\midrule
		Dec 6 & 0.1 & Initial draft\\
		\bottomrule
	\end{tabularx}
\end{table}

\newpage

\pagenumbering{arabic}

This document...

\section{Functional Requirements Evaluation}
	Ori

\section{Nonfunctional Requirements Evaluation}
	Mikhail

	\subsection{Usability}
		Mikhail
	\subsection{Performance}
		Mikhail
	\subsection{etc.}
		Mikhail
	
\section{Comparison to Existing Implementation}
	Ori

\section{Unit Testing}
	Mikhail

\section{Changes Due to Testing}
	Mikhail

\section{Automated Testing}

\subsection{Automated Testing Strategy}
For this project we elected not to use a 3rd party testing library. We made this decision to ease configuration/installation problems and reduce our dependencies, as we judged it would not be necessary. Instead a series of files (labeled test.foobar.cpp) in the repository hold tests, which are run by our custom test runner. These automated tests are run on command by executing the produced executable, or by the continuous integration script run whenever changes are pushed to the central repository. The results of these tests are automatically reported, resulting in a failed or successful build.

\subsection{Specific System Tests}
The following is a list of all system tests in the project.

\begin{center}

\begin{longtable}{ l | l }
\hline
\textbf{Name:} & Amulet Construction\\
\textbf{Initial State:} & None\\
\textbf{Input:} & Coordinate, context value\\
\textbf{Expected Output:} & Amulet object in valid initial state\\
\hline
\textbf{Name:} & Armor Construction 1\\
\textbf{Initial State:} & None\\
\textbf{Input:} & Coordinate\\
\textbf{Expected Output:} & Armor object in valid initial state\\
\hline
\textbf{Name:} & Armor Construction 2\\
\textbf{Initial State:} & None\\
\textbf{Input:} & Coordinate, context value, type value\\
\textbf{Expected Output:} & Armor object in valid initial state\\
\hline
\textbf{Name:} & Armor Identification\\
\textbf{Initial State:} & Cursed Armor\\
\textbf{Input:} & None\\
\textbf{Expected Output:} & Verification that armor is identified\\
\hline
\textbf{Name:} & Armor Identification\\
\textbf{Initial State:} & Cursed Armor\\
\textbf{Input:} & None\\
\textbf{Expected Output:} & Verification that armor is identified\\
\hline
\textbf{Name:} & Armor Curse\\
\textbf{Initial State:} & Cursed Armor\\
\textbf{Input:} & None\\
\textbf{Expected Output:} & Verification that armor is cursed\\
\hline
\textbf{Name:} & Armor Enchantment\\
\textbf{Initial State:} & Cursed Armor\\
\textbf{Input:} & Curse level\\
\textbf{Expected Output:} & Verification that armor enchantment is correct\\
\hline
\textbf{Name:} & Armor Rating\\
\textbf{Initial State:} & Cursed Armor\\
\textbf{Input:} & None\\
\textbf{Expected Output:} & Verification that armor rating is correct\\
\hline
\textbf{Name:} & Coordinate Ordering\\
\textbf{Initial State:} & None\\
\textbf{Input:} & (0,0) coordinate and (1,1) coordinate\\
\textbf{Expected Output:} & Verification that (0,0) < (1,1)\\
\hline
\textbf{Name:} & Coordinate Equality\\
\textbf{Initial State:} & None\\
\textbf{Input:} & Two (0,0) coordinates\\
\textbf{Expected Output:} & Verification that the two inputs are equal\\
\hline
\textbf{Name:} & Coordinate Inequality\\
\textbf{Initial State:} & None\\
\textbf{Input:} & (0,0) coordinate and (1,1) coordinate\\
\textbf{Expected Output:} & Verification that the two inputs are not equal\\
\hline
\textbf{Name:} & Coordinate Addition\\
\textbf{Initial State:} & None\\
\textbf{Input:} & (2,3) coordinate and (1,2) coordinate\\
\textbf{Expected Output:} & (3,5) coordinate\\
\hline
\textbf{Name:} & Coordinate Subtraction\\
\textbf{Initial State:} & None\\
\textbf{Input:} & (2,3) coordinate and (1,2) coordinate\\
\textbf{Expected Output:} & (1,1) coordinate\\
\hline
\textbf{Name:} & Feature Construction\\
\textbf{Initial State:} & None\\
\textbf{Input:} & Symbol, coordinate, visibility, color\\
\textbf{Expected Output:} & Feature object in valid initial state\\
\hline
\textbf{Name:} & Feature Symbol Check\\
\textbf{Initial State:} & Feature with given symbol\\
\textbf{Input:} & Symbol\\
\textbf{Expected Output:} & Verification that feature's symbol matches given\\
\hline
\textbf{Name:} & Feature Invisibility Check\\
\textbf{Initial State:} & Invisible feature\\
\textbf{Input:} & None\\
\textbf{Expected Output:} & Verification that feature is invisible\\
\hline
\textbf{Name:} & Feature Visibility Check\\
\textbf{Initial State:} & Visible feature\\
\textbf{Input:} & None\\
\textbf{Expected Output:} & Verification that feature is visible\\
\hline
\textbf{Name:} & Feature Location Check\\
\textbf{Initial State:} & Feature with given location\\
\textbf{Input:} & Coordinate\\
\textbf{Expected Output:} & Verification that feature's location matches given coordinate\\
\hline
\textbf{Name:} & Food Construction\\
\textbf{Initial State:} & None\\
\textbf{Input:} & Coordinate and context value\\
\textbf{Expected Output:} & Food object in valid initial state\\
\hline
\textbf{Name:} & Food Eating\\
\textbf{Initial State:} & Food and player objects\\
\textbf{Input:} & None\\
\textbf{Expected Output:} & Verification that food has increased the player's food life by an appropriate amount\\
\hline
\textbf{Name:} & GoldPile Construction\\
\textbf{Initial State:} & None\\
\textbf{Input:} & Coordinate, gold amount value\\
\textbf{Expected Output:} & GoldPile object in valid initial state\\
\hline
\textbf{Name:} & GoldPile Quantity Check\\
\textbf{Initial State:} & GoldPile with given amount of gold\\
\textbf{Input:} & Amount of gold value\\
\textbf{Expected Output:} & Verification that gold's amount matches given amount\\
\hline
\textbf{Name:} & Item Construction 1\\
\textbf{Initial State:} & None\\
\textbf{Input:} & Symbol, coordinate, context value, item class specifier, name value, psuedoname value, item type specifier, item stackability value, item throwability value, weight value\\
\textbf{Expected Output:} & Item object in valid initial state\\
\textbf{Name:} & Item Construction 2\\
\textbf{Initial State:} & None\\
\textbf{Input:} & Symbol, coordinate, context value, item class specifier, name value, psuedoname value, item type specifier, item stackability value, item throwability value, weight value\\
\textbf{Expected Output:} & Item object in valid initial state\\
\hline
\textbf{Name:} & Name Vector Check\\
\textbf{Initial State:} & None\\
\textbf{Input:} & Vector of item names\\
\textbf{Expected Output:} & Shuffled vector of item names\\
\hline
\textbf{Name:} & Item Curse Check\\
\textbf{Initial State:} & Uncursed item\\
\textbf{Input:} & None\\
\textbf{Expected Output:} & Verification that item is uncursed\\
\hline
\textbf{Name:} & Item Curse/Effect Check 1\\
\textbf{Initial State:} & Uncursed item to which the cursed effect has been applied\\
\textbf{Input:} & None\\
\textbf{Expected Output:} & Verification that item is cursed\\
\hline
\textbf{Name:} & Item Curse/Effect Check 2\\
\textbf{Initial State:} & Cursed item whose curse effect has been removed\\
\textbf{Input:} & None\\
\textbf{Expected Output:} & Verification that item is uncursed\\
\hline
\textbf{Name:} & Item Unindentified Check\\
\textbf{Initial State:} & Identified item\\
\textbf{Input:} & None\\
\textbf{Expected Output:} & Verification that item is unidentified\\
\hline
\textbf{Name:} & Item Identified Check\\
\textbf{Initial State:} & Unidentified item\\
\textbf{Input:} & None\\
\textbf{Expected Output:} & Verification that item is identified\\
\hline
\textbf{Name:} & Item Display-Name Check 1\\
\textbf{Initial State:} & Unidentified item\\
\textbf{Input:} & Psuedoname\\
\textbf{Expected Output:} & Verification that item's display name matches psuedoname\\
\hline
\textbf{Name:} & Item Display-Name Check 2\\
\textbf{Initial State:} & Identified item\\
\textbf{Input:} & True name\\
\textbf{Expected Output:} & Verification that item's display name matches true name\\
\hline
\textbf{Name:} & ItemZone Containment Check 1\\
\textbf{Initial State:} & ItemZone with 2 items\\
\textbf{Input:} & None\\
\textbf{Expected Output:} & Verification that ItemZone contains the first item\\
\hline
\textbf{Name:} & ItemZone Containment Check 2\\
\textbf{Initial State:} & ItemZone with 2 items\\
\textbf{Input:} & None\\
\textbf{Expected Output:} & Verification that ItemZone contains the second item\\
\hline
\textbf{Name:} & ItemZone Empty Check\\
\textbf{Initial State:} & ItemZone with 2 items\\
\textbf{Input:} & None\\
\textbf{Expected Output:} & Verification that ItemZone is not empty\\
\hline
\textbf{Name:} & ItemZone Size Check\\
\textbf{Initial State:} & ItemZone with 2 items\\
\textbf{Input:} & None\\
\textbf{Expected Output:} & Verification that ItemZone's size is 2\\
\hline
\textbf{Name:} & ItemZone Keybind Check 1\\
\textbf{Initial State:} & ItemZone with 2 items\\
\textbf{Input:} & None\\
\textbf{Expected Output:} & Verification that first item is bound to 'a' key\\
\hline
\textbf{Name:} & ItemZone Keybind Check 2\\
\textbf{Initial State:} & ItemZone with 2 items\\
\textbf{Input:} & None\\
\textbf{Expected Output:} & Verification that second item is bound to 'b' key\\
\hline
\textbf{Name:} & ItemZone Contents Retrieval 1\\
\textbf{Initial State:} & ItemZone with 2 items\\
\textbf{Input:} & None\\
\textbf{Expected Output:} & Item map with exactly 1 copy of first item\\
\hline
\textbf{Name:} & ItemZone Contents Retrieval 2\\
\textbf{Initial State:} & ItemZone with 2 items\\
\textbf{Input:} & None\\
\textbf{Expected Output:} & Item map with exactly 1 copy of second item\\
\hline
\textbf{Name:} & ItemZone Removal\\
\textbf{Initial State:} & ItemZone with 2 items\\
\textbf{Input:} & Removal command\\
\textbf{Expected Output:} & ItemZone with only second item\\
\hline
\textbf{Name:} & ItemZone Keybind Persistence\\
\textbf{Initial State:} & ItemZone with first item removed\\
\textbf{Input:} & None\\
\textbf{Expected Output:} & Verification that second item is still bound to 'b'\\
\hline
\textbf{Name:} & ItemZone Weight Enforcement\\
\textbf{Initial State:} & ItemZone with 1 item\\
\textbf{Input:} & Attempt to add 500 pieces of armor to ItemZone\\
\textbf{Expected Output:} & ItemZone with max-weight worth of armor\\
\hline
\textbf{Name:} & Level Construction\\
\textbf{Initial State:} & None\\
\textbf{Input:} & Depth, player object\\
\textbf{Expected Output:} & Level object in valid initial state\\
\hline
\textbf{Name:} & Level Depth Check\\
\textbf{Initial State:} & Level with given depth\\
\textbf{Input:} & Depth value\\
\textbf{Expected Output:} & Verification that level's depth matches given value\\
\hline
\textbf{Name:} & Level BFSPerp Diagonal Small\\
\textbf{Initial State:} & Empty level object\\
\textbf{Input:} & Pair of coordinates diagonally adjacent\\
\textbf{Expected Output:} & Path between coordinates with expected length, utilizing taxicab movemen\\
\hline
\textbf{Name:} & Level BFSPerp Horizontal\\
\textbf{Initial State:} & Empty level object\\
\textbf{Input:} & Pair of coordinates with equal y-values\\
\textbf{Expected Output:} & Path between coordinates with expected length, utilizing taxicab movemen\\
\hline
\textbf{Name:} & Level BFSPerp Vertical\\
\textbf{Initial State:} & Empty level object\\
\textbf{Input:} & Pair of coordinates with equal x-values\\
\textbf{Expected Output:} & Path between coordinates with expected length, utilizing taxicab movemen\\
\hline
\textbf{Name:} & Level BFSDiag Horizontal\\
\textbf{Initial State:} & Empty level object\\
\textbf{Input:} & Pair of coordinates with equal y-values\\
\textbf{Expected Output:} & Path between coordinates with expected length, utilizing orthogonal movement\\
\hline
\textbf{Name:} & Level BFSDiag Vertical\\
\textbf{Initial State:} & Empty level object\\
\textbf{Input:} & Pair of coordinates with equal x-values\\
\textbf{Expected Output:} & Path between coordinates with expected length, utilizing orthogonal movement\\
\hline
\textbf{Name:} & Level BFSPerp Diagonal\\
\textbf{Initial State:} & Empty level object\\
\textbf{Input:} & Pair of coordinates on diagonal line\\
\textbf{Expected Output:} & Path between coordinates with expected length, utilizing taxicab movement\\
\hline
\textbf{Name:} & Level Starting Position\\
\textbf{Initial State:} & Empty level object\\
\textbf{Input:} & None\\
\textbf{Expected Output:} & Valid starting position coordinate\\
\hline
\textbf{Name:} & Level getAdjPassable\\
\textbf{Initial State:} & Empty level object\\
\textbf{Input:} & Coordinate\\
\textbf{Expected Output:} & List of coordinates orthogonally adjacent to given coordinate\\
\hline
\textbf{Name:} & \\
\textbf{Initial State:} & \\
\textbf{Input:} & \\
\textbf{Expected Output:} & \\
\hline



\textbf{Name:} & \\
\textbf{Initial State:} & \\
\textbf{Input:} & \\
\textbf{Expected Output:} & \\
\hline

\end{longtable}

\end{center}
		
\section{Trace to Requirements}
	Ori

\section{Trace to Modules}		
	Ori

\section{Code Coverage Metrics}
	Ori

\bibliographystyle{plainnat}

\bibliography{SRS}

\end{document}
