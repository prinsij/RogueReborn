\documentclass[12pt, titlepage]{article}

\usepackage{booktabs}
\usepackage{float}
\usepackage{hyperref}
\usepackage{tabularx}
\usepackage{longtable}
\usepackage{indentfirst}
\usepackage{multirow}
\usepackage{tabularx}
\usepackage{listings}
\usepackage[round]{natbib}
\usepackage[usenames, dvipsnames]{color}
\usepackage{tikz}
\usepackage{longtable}



\newcounter{acnum}
\newcommand{\actheacnum}{AC\theacnum}
\newcommand{\acref}[1]{\hyperref[#1]{AC\ref{#1}}}

\newcounter{ucnum}
\newcommand{\uctheucnum}{UC\theucnum}
\newcommand{\uref}[1]{\hyperref[#1]{UC\ref{#1}}}

\newcounter{mnum}
\newcommand{\mhprint}[1]{\addtocounter{mnum}{1} #1 & \textbf{M\themnum}}
\newcommand{\mdprint}[1]{\addtocounter{mnum}{1} #1 (M\themnum)}

\newcolumntype{R}[1]{>{\raggedleft\let\newline\\\arraybackslash\hspace{0em}}p{#1}}

% Setup
\hypersetup{
	colorlinks,
	citecolor=blue,
	filecolor=ForestGreen,
	linkcolor=MidnightBlue,
	urlcolor=blue
}

\lstset{
	basicstyle=\ttfamily\footnotesize
}

\usepackage[round]{natbib}

\title{SE 3XA3: Test Plan\\Rogue Reborn}

\author{Group \#6, Team Rogue++\\\\
	\begin{tabular}{lr}
		Ian Prins & prinsij \\
		Mikhail Andrenkov & andrem5 \\
		Or Almog & almogo
	\end{tabular}
}

\date{Due Wednesday, Dec 7\textsuperscript{st}, 2016}
%% Comments

\usepackage{color}

\newif\ifcomments\commentstrue

\ifcomments
\newcommand{\authornote}[3]{\textcolor{#1}{[#3 ---#2]}}
\newcommand{\todo}[1]{\textcolor{red}{[TODO: #1]}}
\else
\newcommand{\authornote}[3]{}
\newcommand{\todo}[1]{}
\fi

\newcommand{\wss}[1]{\authornote{blue}{SS}{#1}}
\newcommand{\ds}[1]{\authornote{red}{DS}{#1}}
\newcommand{\mj}[1]{\authornote{red}{MSN}{#1}}
\newcommand{\cm}[1]{\authornote{red}{CM}{#1}}
\newcommand{\mh}[1]{\authornote{red}{MH}{#1}}

% team members should be added for each team, like the following
% all comments left by the TAs or the instructor should be addressed
% by a corresponding comment from the Team

\newcommand{\tm}[1]{\authornote{magenta}{Team}{#1}}

\pagenumbering{arabic}

\begin{document}

\maketitle

\newpage
\begin{table}
	\caption{\bf Revision History}
	\begin{tabularx}{\textwidth}{p{3cm}p{2cm}X}
		\toprule {\bf Date} & {\bf Version} & {\bf Notes}\\
		\midrule
		Dec 6 & 0.1 & Initial draft\\
		Dec 6 & 0.2 & Automated Tests Up Through PlayerChar\\
		Dec 6 & 0.3 & Functional requirements eval\\
		\bottomrule
	\end{tabularx}
\end{table}

\newpage
\tableofcontents

\newpage
\listoftables

\newpage
\listoffigures

\newpage
\section{Functional Requirements Evaluation}
	Overall, an evaluation of functional requirements reveals near, if not complete coverage. The tests written for the projects turned out to be quite useful, as many caught bugs or business-errors that would have otherwise gone unnoticed. Those will be discussed below. As for the rest of the functional requirements, many were mundane, general, or crucial enough to have already been satisfied earlier. Those will not be discussed, as their complete satisfaction has already been verified countless times.\\

	The list below refers to each functional requirement by its numerical identifier, as listed in the System Requirements Specification. Please refer to the SRS if any confusion arises due to this.

	\begin{itemize}
		\item[] \textbf{FR.16}: When performing level tests, a strange anomaly led to one test constantly failing. The test revealed that the player, in fact, did not begin at the first level. Due to an off-by-one error and slight miscommunication between developers, the current level depth the player was on was $i$ in some places and $i+1$ in others. As soon as the test revealed this, the problem was remedied globally.
		\item[] \textbf{FR.19}: Whenever the player uncovers a new dungeon level (including the very first level), an algorithm decides on a position in which to place the user initially. This algorithm while appearing flawless, actually had a very slight chance of placing the player in an unreachable location, surrounded by walls, doomed forever. With the automatic tests running thousands upon thousands of simulations, the bug was quickly revealed, and remedied.
		\item[] \textbf{FR.39}: Working with C++ has its benefits, but also its drawbacks. An anomaly in the way C++ handles integers revealed a very serious bug in the code, in which player armor could reach utterly ridiculous values, rendering the player effectively invincible. By simulating every possibility of armor that can be made, this bug was caught and patched. To elaborate, the reason the bug even existed was because an unsigned integer was allowed to be reduced to a negative value, which of course means that it was not reduced to a negative number and instead went to the highest value an integer can be.
	\end{itemize}

\newpage
\section{Nonfunctional Requirements Evaluation}
	Mikhail

	\subsection{Usability}
		Mikhail
	\subsection{Performance}
		Mikhail
	\subsection{etc.}
		Mikhail
	
\newpage
\section{Comparison to Existing Implementation}
	Ori

\newpage
\section{Unit Testing}
	Mikhail

\newpage
\section{Changes Due to Testing}
	Mikhail

\newpage
\section{Automated Testing}

	\subsection{Automated Testing Strategy}
	For this project we elected not to use a 3rd party testing library. We made this decision to ease configuration/installation problems and reduce our dependencies, as we judged it would not be necessary. Instead a series of files (labeled test.foobar.cpp) in the repository hold tests, which are run by our custom test runner. These automated tests are run on command by executing the produced executable, or by the continuous integration script run whenever changes are pushed to the central repository. The results of these tests are automatically reported, resulting in a failed or successful build.

	\subsection{Specific System Tests}
	The following is a list of all system tests in the project.

		\begin{center}

			\begin{longtable}{ l | l }
				\hline
				\textbf{Name:} & Amulet Construction\\
				\textbf{Initial State:} & None\\
				\textbf{Input:} & Coordinate, context value\\
				\textbf{Expected Output:} & Amulet object in valid initial state\\
				\hline
				\textbf{Name:} & Armor Construction 1\\
				\textbf{Initial State:} & None\\
				\textbf{Input:} & Coordinate\\
				\textbf{Expected Output:} & Armor object in valid initial state\\
				\hline
				\textbf{Name:} & Armor Construction 2\\
				\textbf{Initial State:} & None\\
				\textbf{Input:} & Coordinate, context value, type value\\
				\textbf{Expected Output:} & Armor object in valid initial state\\
				\hline
				\textbf{Name:} & Armor Identification\\
				\textbf{Initial State:} & Cursed Armor\\
				\textbf{Input:} & None\\
				\textbf{Expected Output:} & Verification that armor is identified\\
				\hline
				\textbf{Name:} & Armor Identification\\
				\textbf{Initial State:} & Cursed Armor\\
				\textbf{Input:} & None\\
				\textbf{Expected Output:} & Verification that armor is identified\\
				\hline
				\textbf{Name:} & Armor Curse\\
				\textbf{Initial State:} & Cursed Armor\\
				\textbf{Input:} & None\\
				\textbf{Expected Output:} & Verification that armor is cursed\\
				\hline
				\textbf{Name:} & Armor Enchantment\\
				\textbf{Initial State:} & Cursed Armor\\
				\textbf{Input:} & Curse level\\
				\textbf{Expected Output:} & Verification that armor enchantment is correct\\
				\hline
				\textbf{Name:} & Armor Rating\\
				\textbf{Initial State:} & Cursed Armor\\
				\textbf{Input:} & None\\
				\textbf{Expected Output:} & Verification that armor rating is correct\\
				\hline
				\textbf{Name:} & Coordinate Ordering\\
				\textbf{Initial State:} & None\\
				\textbf{Input:} & (0,0) coordinate and (1,1) coordinate\\
				\textbf{Expected Output:} & Verification that (0,0) < (1,1)\\
				\hline
				\textbf{Name:} & Coordinate Equality\\
				\textbf{Initial State:} & None\\
				\textbf{Input:} & Two (0,0) coordinates\\
				\textbf{Expected Output:} & Verification that the two inputs are equal\\
				\hline
				\textbf{Name:} & Coordinate Inequality\\
				\textbf{Initial State:} & None\\
				\textbf{Input:} & (0,0) coordinate and (1,1) coordinate\\
				\textbf{Expected Output:} & Verification that the two inputs are not equal\\
				\hline
				\textbf{Name:} & Coordinate Addition\\
				\textbf{Initial State:} & None\\
				\textbf{Input:} & (2,3) coordinate and (1,2) coordinate\\
				\textbf{Expected Output:} & (3,5) coordinate\\
				\hline
				\textbf{Name:} & Coordinate Subtraction\\
				\textbf{Initial State:} & None\\
				\textbf{Input:} & (2,3) coordinate and (1,2) coordinate\\
				\textbf{Expected Output:} & (1,1) coordinate\\
				\hline
				\textbf{Name:} & Feature Construction\\
				\textbf{Initial State:} & None\\
				\textbf{Input:} & Symbol, coordinate, visibility, color\\
				\textbf{Expected Output:} & Feature object in valid initial state\\
				\hline
				\textbf{Name:} & Feature Symbol Check\\
				\textbf{Initial State:} & Feature with given symbol\\
				\textbf{Input:} & Symbol\\
				\textbf{Expected Output:} & Verification that feature's symbol matches given\\
				\hline
				\textbf{Name:} & Feature Invisibility Check\\
				\textbf{Initial State:} & Invisible feature\\
				\textbf{Input:} & None\\
				\textbf{Expected Output:} & Verification that feature is invisible\\
				\hline
				\textbf{Name:} & Feature Visibility Check\\
				\textbf{Initial State:} & Visible feature\\
				\textbf{Input:} & None\\
				\textbf{Expected Output:} & Verification that feature is visible\\
				\hline
				\textbf{Name:} & Feature Location Check\\
				\textbf{Initial State:} & Feature with given location\\
				\textbf{Input:} & Coordinate\\
				\textbf{Expected Output:} & Verification that feature's location matches given coordinate\\
				\hline
				\textbf{Name:} & Food Construction\\
				\textbf{Initial State:} & None\\
				\textbf{Input:} & Coordinate and context value\\
				\textbf{Expected Output:} & Food object in valid initial state\\
				\hline
				\textbf{Name:} & Food Eating\\
				\textbf{Initial State:} & Food and player objects\\
				\textbf{Input:} & None\\
				\textbf{Expected Output:} & Verification that food has increased the player's food life by an appropriate amount\\
				\hline
				\textbf{Name:} & GoldPile Construction\\
				\textbf{Initial State:} & None\\
				\textbf{Input:} & Coordinate, gold amount value\\
				\textbf{Expected Output:} & GoldPile object in valid initial state\\
				\hline
				\textbf{Name:} & GoldPile Quantity Check\\
				\textbf{Initial State:} & GoldPile with given amount of gold\\
				\textbf{Input:} & Amount of gold value\\
				\textbf{Expected Output:} & Verification that gold's amount matches given amount\\
				\hline
				\textbf{Name:} & Item Construction 1\\
				\textbf{Initial State:} & None\\
				\textbf{Input:} & Symbol, coordinate, context value, item class specifier, name value, psuedoname value, item type specifier, item stackability value, item throwability value, weight value\\
				\textbf{Expected Output:} & Item object in valid initial state\\
				\textbf{Name:} & Item Construction 2\\
				\textbf{Initial State:} & None\\
				\textbf{Input:} & Symbol, coordinate, context value, item class specifier, name value, psuedoname value, item type specifier, item stackability value, item throwability value, weight value\\
				\textbf{Expected Output:} & Item object in valid initial state\\
				\hline
				\textbf{Name:} & Name Vector Check\\
				\textbf{Initial State:} & None\\
				\textbf{Input:} & Vector of item names\\
				\textbf{Expected Output:} & Shuffled vector of item names\\
				\hline
				\textbf{Name:} & Item Curse Check\\
				\textbf{Initial State:} & Uncursed item\\
				\textbf{Input:} & None\\
				\textbf{Expected Output:} & Verification that item is uncursed\\
				\hline
				\textbf{Name:} & Item Curse/Effect Check 1\\
				\textbf{Initial State:} & Uncursed item to which the cursed effect has been applied\\
				\textbf{Input:} & None\\
				\textbf{Expected Output:} & Verification that item is cursed\\
				\hline
				\textbf{Name:} & Item Curse/Effect Check 2\\
				\textbf{Initial State:} & Cursed item whose curse effect has been removed\\
				\textbf{Input:} & None\\
				\textbf{Expected Output:} & Verification that item is uncursed\\
				\hline
				\textbf{Name:} & Item Unindentified Check\\
				\textbf{Initial State:} & Identified item\\
				\textbf{Input:} & None\\
				\textbf{Expected Output:} & Verification that item is unidentified\\
				\hline
				\textbf{Name:} & Item Identified Check\\
				\textbf{Initial State:} & Unidentified item\\
				\textbf{Input:} & None\\
				\textbf{Expected Output:} & Verification that item is identified\\
				\hline
				\textbf{Name:} & Item Display-Name Check 1\\
				\textbf{Initial State:} & Unidentified item\\
				\textbf{Input:} & Psuedoname\\
				\textbf{Expected Output:} & Verification that item's display name matches psuedoname\\
				\hline
				\textbf{Name:} & Item Display-Name Check 2\\
				\textbf{Initial State:} & Identified item\\
				\textbf{Input:} & True name\\
				\textbf{Expected Output:} & Verification that item's display name matches true name\\
				\hline
				\textbf{Name:} & ItemZone Containment Check 1\\
				\textbf{Initial State:} & ItemZone with 2 items\\
				\textbf{Input:} & None\\
				\textbf{Expected Output:} & Verification that ItemZone contains the first item\\
				\hline
				\textbf{Name:} & ItemZone Containment Check 2\\
				\textbf{Initial State:} & ItemZone with 2 items\\
				\textbf{Input:} & None\\
				\textbf{Expected Output:} & Verification that ItemZone contains the second item\\
				\hline
				\textbf{Name:} & ItemZone Empty Check\\
				\textbf{Initial State:} & ItemZone with 2 items\\
				\textbf{Input:} & None\\
				\textbf{Expected Output:} & Verification that ItemZone is not empty\\
				\hline
				\textbf{Name:} & ItemZone Size Check\\
				\textbf{Initial State:} & ItemZone with 2 items\\
				\textbf{Input:} & None\\
				\textbf{Expected Output:} & Verification that ItemZone's size is 2\\
				\hline
				\textbf{Name:} & ItemZone Keybind Check 1\\
				\textbf{Initial State:} & ItemZone with 2 items\\
				\textbf{Input:} & None\\
				\textbf{Expected Output:} & Verification that first item is bound to 'a' key\\
				\hline
				\textbf{Name:} & ItemZone Keybind Check 2\\
				\textbf{Initial State:} & ItemZone with 2 items\\
				\textbf{Input:} & None\\
				\textbf{Expected Output:} & Verification that second item is bound to 'b' key\\
				\hline
				\textbf{Name:} & ItemZone Contents Retrieval 1\\
				\textbf{Initial State:} & ItemZone with 2 items\\
				\textbf{Input:} & None\\
				\textbf{Expected Output:} & Item map with exactly 1 copy of first item\\
				\hline
				\textbf{Name:} & ItemZone Contents Retrieval 2\\
				\textbf{Initial State:} & ItemZone with 2 items\\
				\textbf{Input:} & None\\
				\textbf{Expected Output:} & Item map with exactly 1 copy of second item\\
				\hline
				\textbf{Name:} & ItemZone Removal\\
				\textbf{Initial State:} & ItemZone with 2 items\\
				\textbf{Input:} & Removal command\\
				\textbf{Expected Output:} & ItemZone with only second item\\
				\hline
				\textbf{Name:} & ItemZone Keybind Persistence\\
				\textbf{Initial State:} & ItemZone with first item removed\\
				\textbf{Input:} & None\\
				\textbf{Expected Output:} & Verification that second item is still bound to 'b'\\
				\hline
				\textbf{Name:} & ItemZone Weight Enforcement\\
				\textbf{Initial State:} & Empty ItemZone\\
				\textbf{Input:} & Attempt to add 500 pieces of armor to ItemZone\\
				\textbf{Expected Output:} & ItemZone with max-weight worth of armor\\
				\hline
				\textbf{Name:} & Level Construction\\
				\textbf{Initial State:} & None\\
				\textbf{Input:} & Depth, player object\\
				\textbf{Expected Output:} & Level object in valid initial state\\
				\hline
				\textbf{Name:} & Level Depth Check\\
				\textbf{Initial State:} & Level with given depth\\
				\textbf{Input:} & Depth value\\
				\textbf{Expected Output:} & Verification that level's depth matches given value\\
				\hline
				\textbf{Name:} & Level BFSPerp Diagonal Small\\
				\textbf{Initial State:} & Empty level object\\
				\textbf{Input:} & Pair of coordinates diagonally adjacent\\
				\textbf{Expected Output:} & Path between coordinates with expected length, utilizing taxicab movemen\\
				\hline
				\textbf{Name:} & Level BFSPerp Horizontal\\
				\textbf{Initial State:} & Empty level object\\
				\textbf{Input:} & Pair of coordinates with equal y-values\\
				\textbf{Expected Output:} & Path between coordinates with expected length, utilizing taxicab movemen\\
				\hline
				\textbf{Name:} & Level BFSPerp Vertical\\
				\textbf{Initial State:} & Empty level object\\
				\textbf{Input:} & Pair of coordinates with equal x-values\\
				\textbf{Expected Output:} & Path between coordinates with expected length, utilizing taxicab movemen\\
				\hline
				\textbf{Name:} & Level BFSDiag Horizontal\\
				\textbf{Initial State:} & Empty level object\\
				\textbf{Input:} & Pair of coordinates with equal y-values\\
				\textbf{Expected Output:} & Path between coordinates with expected length, utilizing orthogonal movement\\
				\hline
				\textbf{Name:} & Level BFSDiag Vertical\\
				\textbf{Initial State:} & Empty level object\\
				\textbf{Input:} & Pair of coordinates with equal x-values\\
				\textbf{Expected Output:} & Path between coordinates with expected length, utilizing orthogonal movement\\
				\hline
				\textbf{Name:} & Level BFSPerp Diagonal\\
				\textbf{Initial State:} & Empty level object\\
				\textbf{Input:} & Pair of coordinates on diagonal line\\
				\textbf{Expected Output:} & Path between coordinates with expected length, utilizing taxicab movement\\
				\hline
				\textbf{Name:} & Level Starting Position\\
				\textbf{Initial State:} & Empty level object\\
				\textbf{Input:} & None\\
				\textbf{Expected Output:} & Valid starting position coordinate\\
				\hline
				\textbf{Name:} & Level getAdjPassable\\
				\textbf{Initial State:} & Empty level object\\
				\textbf{Input:} & Coordinate\\
				\textbf{Expected Output:} & List of coordinates orthogonally adjacent to given coordinate\\
				\hline
				\textbf{Name:} & Level Path Generation\\
				\textbf{Initial State:} & Player object and generated level\\
				\textbf{Input:} & Series of path requests between random coordinates\\
				\textbf{Expected Output:} & Valid paths between locations\\
				\hline
				\textbf{Name:} & Level Connectedness\\
				\textbf{Initial State:} & Player object and generated level\\
				\textbf{Input:} & Series of path requests between all rooms in the level\\
				\textbf{Expected Output:} & Valid paths between each room\\
				\hline
				\textbf{Name:} & Level Staircase Check\\
				\textbf{Initial State:} & Player object and generated level\\
				\textbf{Input:} & None\\
				\textbf{Expected Output:} & Verification that level contains a staircase\\
				\hline
				\textbf{Name:} & Level GoldPile Check\\
				\textbf{Initial State:} & Player object and generated level\\
				\textbf{Input:} & None\\
				\textbf{Expected Output:} & Verification that level contains at least one goldpile\\
				\hline
				\textbf{Name:} & Monster Construction\\
				\textbf{Initial State:} & None\\
				\textbf{Input:} & Symbol, coordinate, armor value, HP value, exp value, level value, maxHP value, name value\\
				\textbf{Expected Output:} & Monster object in valid initial state\\
				\hline
				\textbf{Name:} & Dice-Math 1\\
				\textbf{Initial State:} & None\\
				\textbf{Input:} & 1 1-sided die\\
				\textbf{Expected Output:} & Sum of values of 1\\
				\hline
				\textbf{Name:} & Dice-Math 2\\
				\textbf{Initial State:} & None\\
				\textbf{Input:} & 2 1-sided die\\
				\textbf{Expected Output:} & Sum of values of 2\\
				\hline
				\textbf{Name:} & Dice-Math 3\\
				\textbf{Initial State:} & None\\
				\textbf{Input:} & 1 2-sided die\\
				\textbf{Expected Output:} & 1 <= Sum of values <= 2\\
				\hline
				\textbf{Name:} & Dice-Math 4\\
				\textbf{Initial State:} & None\\
				\textbf{Input:} & 3 4-sided die\\
				\textbf{Expected Output:} & 3 <= Sum of values <= 12\\
				\hline
				\textbf{Name:} & Mob Armor Check\\
				\textbf{Initial State:} & Mob object\\
				\textbf{Input:} & None\\
				\textbf{Expected Output:} & Verification mob armor is in valid range\\
				\hline
				\textbf{Name:} & Mob HP Check 1\\
				\textbf{Initial State:} & Mob with given HP value\\
				\textbf{Input:} & HP value\\
				\textbf{Expected Output:} & Verification mob has correct HP value\\
				\hline
				\textbf{Name:} & Mob MaxHP Check\\
				\textbf{Initial State:} & Mob with given MaxHP value\\
				\textbf{Input:} & MaxHP value\\
				\textbf{Expected Output:} & Verification mob has correct MaxHP value\\
				\hline
				\textbf{Name:} & Mob Level Check\\
				\textbf{Initial State:} & Mob with given level value\\
				\textbf{Input:} & Level value\\
				\textbf{Expected Output:} & Verification mob has correct level value\\
				\hline
				\textbf{Name:} & Mob Location Check\\
				\textbf{Initial State:} & Mob with given location\\
				\textbf{Input:} & Coordinate\\
				\textbf{Expected Output:} & Verification mob has correct location\\
				\hline
				\textbf{Name:} & Mob Name Check\\
				\textbf{Initial State:} & Mob with given name\\
				\textbf{Input:} & Name value\\
				\textbf{Expected Output:} & Verification mob has correct name\\
				\hline
				\textbf{Name:} & Mob setMaxHP\\
				\textbf{Initial State:} & Mob with default MaxHP\\
				\textbf{Input:} & setMaxHP command with MaxHP value\\
				\textbf{Expected Output:} & mob with given MaxHP value\\
				\hline
				\textbf{Name:} & Mob setcurrentHP\\
				\textbf{Initial State:} & Mob with default currentHP\\
				\textbf{Input:} & setCurrentHP command with currentHP value\\
				\textbf{Expected Output:} & mob with given currentHP value\\
				\hline
				\textbf{Name:} & Mob Dead Check 1\\
				\textbf{Initial State:} & Living Mob object\\
				\textbf{Input:} & None\\
				\textbf{Expected Output:} & Verification mob is alive\\
				\hline
				\textbf{Name:} & Mob HP Check 2\\
				\textbf{Initial State:} & Living Mob object\\
				\textbf{Input:} & Hit command for >>> mob's current HP\\
				\textbf{Expected Output:} & Verification mob has HP <= 0\\
				\hline
				\textbf{Name:} & Mob Dead Check 2\\
				\textbf{Initial State:} & Dead mob object\\
				\textbf{Input:} & None\\
				\textbf{Expected Output:} & Verification mob is dead\\
				\hline
				\textbf{Name:} & Monster Construction\\
				\textbf{Initial State:} & None\\
				\textbf{Input:} & Symbol, coordinate\\
				\textbf{Expected Output:} & Monster object in valid initial state\\
				\hline
				\textbf{Name:} & Monster Flag/Invisibility\\
				\textbf{Initial State:} & Visible monster object\\
				\textbf{Input:} & SetFlag command to make monster invisible\\
				\textbf{Expected Output:} & Invisible monster object\\
				\hline
				\textbf{Name:} & Monster Aggrevate\\
				\textbf{Initial State:} & Idling, sleeping monster object\\
				\textbf{Input:} & Aggrevate command\\
				\textbf{Expected Output:} & Awake, chasing monster object\\
				\hline
				\textbf{Name:} & Monster Damage Calculation\\
				\textbf{Initial State:} & Monster object\\
				\textbf{Input:} & calculateDamage command\\
				\textbf{Expected Output:} & Correct amount of damage\\
				\hline
				\textbf{Name:} & Monster Hit Chance\\
				\textbf{Initial State:} & Monster and player objects\\
				\textbf{Input:} & calculateHitChange command\\
				\textbf{Expected Output:} & Hit chance in valid range\\
				\hline
				\textbf{Name:} & Monster Armor Check\\
				\textbf{Initial State:} & Monster object\\
				\textbf{Input:} & None\\
				\textbf{Expected Output:} & Verification that monster armor is in valid range\\
				\hline
				\textbf{Name:} & Invisible Monster Name Check\\
				\textbf{Initial State:} & Invisible uonster object\\
				\textbf{Input:} & None\\
				\textbf{Expected Output:} & Verification monster has hidden name\\
				\hline
				\textbf{Name:} & Visible Monster Name Check\\
				\textbf{Initial State:} & Invisible monster object\\
				\textbf{Input:} & RemoveFlag command to make monster invisible\\
				\textbf{Expected Output:} & Verification monster has real name\\
				\hline
				\textbf{Name:} & Monster Symbol/Level Association\\
				\textbf{Initial State:} & None\\
				\textbf{Input:} & Depth value\\
				\textbf{Expected Output:} & Set of symbols for monsters that are valid candidates for given depth\\
				\hline
				\textbf{Name:} & Monster Symbol/Treasure/Level Association\\
				\textbf{Initial State:} & None\\
				\textbf{Input:} & Depth value\\
				\textbf{Expected Output:} & Set of symbols for monsters that are valid candidates for given depth for a treasure room\\
				\hline
				\textbf{Name:} & PlayerChar Initial Amulet Check\\
				\textbf{Initial State:} & Just initialized playerchar object\\
				\textbf{Input:} & None\\
				\textbf{Expected Output:} & Verification the game does not believe the player has the amulet\\
				\hline
				\textbf{Name:} & PlayerChar Initial HP Check\\
				\textbf{Initial State:} & Just initialized playerchar object\\
				\textbf{Input:} & None\\
				\textbf{Expected Output:} & Verification playerchar has full hp\\
				\hline
				\textbf{Name:} & PlayerChar Level-Up Exp\\
				\textbf{Initial State:} & Playerchar object at initial level\\
				\textbf{Input:} & Exp input into playerchar object\\
				\textbf{Expected Output:} & Playerchar object with increased level\\
				\hline
				\textbf{Name:} & PlayerChar Level-Up Manual\\
				\textbf{Initial State:} & Playerchar object\\
				\textbf{Input:} & Level-up command\\
				\textbf{Expected Output:} & Playerchar object with increased level\\
				\hline
				\textbf{Name:} & PlayerChar Damage\\
				\textbf{Initial State:} & Playerchar object at full hp\\
				\textbf{Input:} & Series of damage commands applied to playerchar object\\
				\textbf{Expected Output:} & Playerchar object with less than full hp\\
				\hline
				\textbf{Name:} & PlayerChar UnArmed 1\\
				\textbf{Initial State:} & Unarmed playerchar object\\
				\textbf{Input:} & calculateDamage command\\
				\textbf{Expected Output:} & 0 damage value\\
				\hline
				\textbf{Name:} & PlayerChar Armed\\
				\textbf{Initial State:} & Playerchar object armed with weapon\\
				\textbf{Input:} & calculateDamage command\\
				\textbf{Expected Output:} & Damage value > 0\\
				\hline
				\textbf{Name:} & PlayerChar Stow Weapon\\
				\textbf{Initial State:} & Playerchar object armed with uncursed weapon\\
				\textbf{Input:} & removeWeapon command\\
				\textbf{Expected Output:} & PlayerChar object unarmed\\
				\hline
				\textbf{Name:} & PlayerChar UnArmed 2\\
				\textbf{Initial State:} & Armed playerchar object\\
				\textbf{Input:} & removeWeapon command, then calculateDamage\\
				\textbf{Expected Output:} & 0 damage value\\
				\hline
				\textbf{Name:} & PlayerChar Remove Non-Armor\\
				\textbf{Initial State:} & Playerchar object with no armor\\
				\textbf{Input:} & removeArmor command\\
				\textbf{Expected Output:} & Boolean indicating failure to remove armor\\
				\hline
				\textbf{Name:} & PlayerChar Remove Armor\\
				\textbf{Initial State:} & Playerchar object with uncursed armor\\
				\textbf{Input:} & removeArmor command\\
				\textbf{Expected Output:} & Playerchar object without armor\\
				\hline



				\textbf{Name:} & \\
				\textbf{Initial State:} & \\
				\textbf{Input:} & \\
				\textbf{Expected Output:} & \\
				\hline

			\end{longtable}

		\end{center}

\newpage
\section{Trace to Requirements}

	The following table maps each implemented test file to a set of functional and non-functional requirements 

	\begin{table}[h!]
		\caption{\bf Test-Requirement Trace}
		\label{TblMH}
		\bigskip
		\centering
		\begin{longtable}{lr}
			\hline
			File & Related Requirement(s)\\
			\hline
			test.amulet.cpp 		& FR\\
			test.armor.cpp 			& FR\\
			test.coord.cpp 			& FR\\
			test.feature.cpp 		& FR\\
			test.food.cpp 			& FR\\
			test.goldpile.cpp 		& FR\\
			test.item.cpp 			& FR\\
			test.itemzone.cpp 		& FR\\
			test.level.cpp 			& FR\\
			test.levelgen.cpp 		& FR\\
			test.main.cpp 			& FR\\
			test.mob.cpp 			& FR\\
			test.monster.cpp 		& FR\\
			test.playerchar.cpp 	& FR\\
			test.potion.cpp 		& FR\\
			test.ring.cpp 			& FR\\
			test.room.cpp 			& FR\\
			test.scroll.cpp 		& FR\\
			test.stairs.cpp 		& FR\\
			test.terrain.cpp 		& FR\\
			test.testable.cpp 		& FR\\
			test.testable.h 		& FR\\
			test.trap.cpp 			& FR\\
			test.tunnel.cpp 		& FR\\
			test.uistate.cpp 		& FR\\
			test.wand.cpp 			& FR\\
			test.weapon.cpp 		& FR\\
			\hline
		\end{longtable}
	\end{table}

\newpage
\section{Trace to Modules}

	The following table re-iterates the modules of the project, along with their respective domain and module ID. The module IDs are used to refer to modules in the trace. More about the modules can be found in the Module Guide.\\

	\begin{table}[h!]
		\caption{\bf Module Hierarchy}
		\label{TblMH}
		\bigskip
		\centering
		\begin{tabular}{p{0.3\textwidth} p{0.5\textwidth} R{0.10\textwidth}}
			\toprule
			\textbf{Level 1} & \multicolumn{2}{l}{\textbf{Level 2}}\\
			\midrule

			\multirow{2}{0.3\textwidth}{Hardware-Hiding Module}
			& \mhprint{BasicIO}\\
			& \mhprint{Doryen}\\
			& \mhprint{Input Format}\\
			\midrule

			\multirow{6}{0.3\textwidth}{Behaviour-Hiding Module}
			& \mhprint{External}\\
			& \mhprint{Item}\\
			& \mhprint{Level}\\
			& \mhprint{LevelGen}\\
			& \mhprint{MainMenu}\\  
			& \mhprint{Monster}\\
			& \mhprint{PlayerChar}\\
			& \mhprint{RipScreen}\\
			& \mhprint{PlayState}\\
			& \mhprint{UIState}\\
			\midrule

			\multirow{9}{0.3\textwidth}{Software Decision Module}
			& \mhprint{Coord}\\
			& \mhprint{Feature}\\
			& \mhprint{ItemZone}\\          
			& \mhprint{MasterController}\\            
			& \mhprint{Mob}\\
			& \mhprint{Random}\\
			& \mhprint{Terrain}\\
			\bottomrule
		\end{tabular}
		 \setcounter{mnum}{0}
	\end{table}

	\newpage

	The following table maps test files, which implement tests, to specific modules, given by their IDs.\\

	\begin{table}[h!]
		\caption{\bf Test-Module Trace}
		\label{TblMH}
		\bigskip
		\centering
		\begin{longtable}{lr}
			\hline
			File & Related Module(s)\\
			\hline
			test.amulet.cpp 		& M7, M12, M13\\
			test.armor.cpp 			& M5, M10, M18\\
			test.coord.cpp 			& M2, M5, M6, M7, M14, M19\\
			test.feature.cpp 		& M5, M15, M16, M10\\
			test.food.cpp 			& M5, M6, M7, M10, M12\\
			test.goldpile.cpp 		& M5, M6, M7, M9, M10, M15, M16\\
			test.item.cpp 			& M5, M15\\
			test.itemzone.cpp 		& M5, M6, M14, M15, M16\\
			test.level.cpp 			& M5, M6, M9, M10, M14, M15, M19\\
			test.levelgen.cpp 		& M5, M6, M9, M14, M15, M19, M20\\
			test.main.cpp 			& None (Puts everything together)\\
			test.mob.cpp 			& M9, M10, M12, M13, M14, M18\\
			test.monster.cpp 		& M9, M14, M18\\
			test.playerchar.cpp 	& M5, M6, M10, M11, M12, M13, M14, M15, M16, M17, M18\\
			test.potion.cpp 		& M5, M6, M7, M9, M10, M15, M16\\
			test.ring.cpp 			& M5, M6, M7, M9, M10, M15, M16\\
			test.room.cpp 			& M6, M7, M14, M19\\
			test.scroll.cpp 		& M5, M6, M7, M9, M10, M15, M16\\
			test.stairs.cpp 		& M7, M15, M17, M20\\
			test.terrain.cpp 		& M6, M7, M19, M20\\
			test.testable.cpp 		& Defines test-suite\\
			test.testable.h 		& Defines test-suite\\
			test.trap.cpp 			& M6, M7, M10, M13, M15\\
			test.tunnel.cpp 		& M6, M5, M14\\
			test.uistate.cpp 		& M4, M8, M11, M12, M13, M17\\
			test.wand.cpp 			& M5, M6, M7, M9, M10, M15, M16\\
			test.weapon.cpp 		& M5, M6, M7, M9, M10, M15, M16\\
			\hline
		\end{longtable}
	\end{table}



\newpage
\section{Code Coverage Metrics}
	Cross-referencing the test-module trace above with the module-requirements trace given in the Module Guide, it is possible to determine exactly which functional and non-functional requirements were satisfied with the test cases we created.\\

	As can be expected, near \textbf{complete coverage} of both functional and non-functional requirements in achieved. Except for a few non-functional requirements, the modules reflected in the test cases offer a complete coverage of the requirements. This is good news, as it means that, while perhaps indirectly, the specified test cases 

\bibliographystyle{plainnat}

\bibliography{SRS}

\end{document}
