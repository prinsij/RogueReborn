\documentclass{article}

\usepackage{booktabs}
\usepackage{tabularx}

\title{SE 3XA3: Problem Statement\\Rogue Reborn}

\author{Group \#6, Team Rogue++\\\\
	\begin{tabular} {l l}
		Ian Prins & prinsij \\
		Mikhail Andrenkov & andrem5 \\
		Or Almog  & almogo
	\end{tabular}
}

\date{Friday, September 23, 2016}

%% Comments

\usepackage{color}

\newif\ifcomments\commentstrue

\ifcomments
\newcommand{\authornote}[3]{\textcolor{#1}{[#3 ---#2]}}
\newcommand{\todo}[1]{\textcolor{red}{[TODO: #1]}}
\else
\newcommand{\authornote}[3]{}
\newcommand{\todo}[1]{}
\fi

\newcommand{\wss}[1]{\authornote{blue}{SS}{#1}}
\newcommand{\ds}[1]{\authornote{red}{DS}{#1}}
\newcommand{\mj}[1]{\authornote{red}{MSN}{#1}}
\newcommand{\cm}[1]{\authornote{red}{CM}{#1}}
\newcommand{\mh}[1]{\authornote{red}{MH}{#1}}

% team members should be added for each team, like the following
% all comments left by the TAs or the instructor should be addressed
% by a corresponding comment from the Team

\newcommand{\tm}[1]{\authornote{magenta}{Team}{#1}}


\begin{document}

\begin{table}[hp]
	\caption{Revision History} \label{TblRevisionHistory}
	\begin{tabularx}{\textwidth}{llX}
		\toprule
		\textbf{Date} & \textbf{Developer(s)} & \textbf{Change}\\
		\midrule
		09/22/16 & Or Almog & Made \& compiled .tex\\
		09/22/16 & Mikhail Andrenkov & Edited and Proofread Content\\
		09/23/16 & Mikhail Andrenkov & Removed Unnecessary Technical Details\\
		... & ... & ...\\
		\bottomrule
	\end{tabularx}
\end{table}

\newpage

\maketitle
From the humble beginnings of \textit{Pong} and \textit{Pacman}, the video game industry has evolved to define modern entertainment by constantly pushing the limits of hardware and software. The technology behind contemporary video games encompasses a large variety of academic subjects and can be as sophisticated and rigorous as any other traditional software project. Although it is easy to forget that modern developers stand on the shoulders of giants, it is important to remember that even classic video games were considered immense feats of technical accomplishment in their time.  As such, the Rogue Reborn project aims to re-create the legendary \textit{Rogue} (1980): an iconic video game best known for pioneering the ``roguelike'' game genre.\\

In short, the problem to be solved by the Rogue++ team is the issue of boredom.  In a society where information and amusement is omnipresent, it is no secret that people generally have less energy to expend on duller subjects that are ultimately more important than their more lively rivals.  To lighten the mood during the more boring times, a simple, well-designed video game can serve to help people relax, engage, and enjoy themselves.  This form of entertainment can also increase future productivity and creativity by stimulating the brain during originative problem-solving tasks.  By designing a modern port of the original \textit{Rogue}, people today can have access to a timeless and invaluable entertainment piece that will run on almost any personal-use desktop environment.\\

The need for a \textit{Rogue} re-make is also characterized by the legacy software design techniques that were used in the original release; these practices made the original source difficult to read, compile, understand, and modify.  The aim of this project is to rewrite Rogue in a modern programming language, as well as provide a full set of test cases and documentation to accompany the deliverable.  The new product will be functionally equivalent to the original, although the new version may contain several improvements to the non-functional qualities of the game.\\

One of the primary stakeholders of the Rogue Reborn project is everyone who will be playing the game (most likely players who are familiar with the original \textit{Rogue} game or its successors) for the rest and relaxation reasons mentioned above.  Given that \textit{Rogue} is the origin for the roguelike genre, it also offers historical and nostalgic value for some of the more veteran stakeholders.  Another group of interested stakeholders includes game developers, since it is common for programmers to look at successful past products for insights while developing their own projects.  A modern, well-documented version of \textit{Rogue} can prove to be quite valuable for aspiring game developers.  Finally, Professor Smith, the 3XA3 TAs, and the project playtesters are an additional group of stakeholders, as they are responsible for evaluating the project and have a significant impact on the final deliverable.  On the whole, the Rogue Reborn project should allow interested stakeholders to clearly examine the inner workings of the game, enable easy modification and extension of the game, and provide direct entertainment value much like the original game.

\end{document}
