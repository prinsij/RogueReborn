\documentclass{article}

\usepackage{booktabs}
\usepackage{tabularx}

\title{SE 3XA3: Problem Statement\\Rogue Reborn}

\author{Team number: 6
		\\ Team name: TeamSix
		\\ Ian Prins (prinsij)
		\\ Mikhail Andrenkov (andrem5)
		\\ Or Almog (almogo)
}

\date{23/09/2016}

%% Comments

\usepackage{color}

\newif\ifcomments\commentstrue

\ifcomments
\newcommand{\authornote}[3]{\textcolor{#1}{[#3 ---#2]}}
\newcommand{\todo}[1]{\textcolor{red}{[TODO: #1]}}
\else
\newcommand{\authornote}[3]{}
\newcommand{\todo}[1]{}
\fi

\newcommand{\wss}[1]{\authornote{blue}{SS}{#1}}
\newcommand{\ds}[1]{\authornote{red}{DS}{#1}}
\newcommand{\mj}[1]{\authornote{red}{MSN}{#1}}
\newcommand{\cm}[1]{\authornote{red}{CM}{#1}}
\newcommand{\mh}[1]{\authornote{red}{MH}{#1}}

% team members should be added for each team, like the following
% all comments left by the TAs or the instructor should be addressed
% by a corresponding comment from the Team

\newcommand{\tm}[1]{\authornote{magenta}{Team}{#1}}


\begin{document}

\begin{table}[hp]
\caption{Revision History} \label{TblRevisionHistory}
\begin{tabularx}{\textwidth}{llX}
\toprule
\textbf{Date} & \textbf{Developer(s)} & \textbf{Change}\\
\midrule
Sep 22 & Or Almog & Made \& compiled .tex\\
... & ... & ...\\
\bottomrule
\end{tabularx}
\end{table}

\newpage

\maketitle


From the humble beginnings of Pong and Pacman, the video game industry has evolved to define modern entertainment by constantly pushing the limits of hardware and software. The technology behind contemporary video games encompasses a large variety of academic subjects and can be as sophisticated and rigorous as any other traditional software project. Although it is easy to forget that modern developers stand on the shoulders of giants, it is important to remember that even classic video games were considered immense feats of technical accomplishment in their time. The primary goal of the "Rogue Reborn" project is to re-create what is arguably one of the most iconic video games of all time: Rogue.\\


The problem being solved is the problem of boredom. Effective recreational time is important to overall productivity, and as such, well-designed video games provide societal utility. The original Rogue (1980) is a classic video game that is best known for pioneering the "roguelike" game genre, which was named after the original game. However, the original source is difficult to read, not benefiting from modern software design techniques. This makes it difficult to compile, understand, or modify the original source. This project will rewrite Rogue in a modern language (C++) using modern techniques including OOP, along with a full set of tests and documentation. The rewrite should be a functionally identical update to the original.\\


The software will be developed for a Linux desktop environment, particularly one for personal use. As the rewrite will use text-based graphics like the original, the environment does not exclude older/slower hardware. The library in use (libtcod) has support for Windows environments, so a hypothetical windows port would involve only minimal changes. Unlike the original, the rewrite will use a graphical/emulated console. This enhances portability, eases development, and allows for greater accessibility.\\


Stakeholders include players of the original Rogue game as well as players and developers of successor games, who may be interested in playing/developing from the original. Game developers often look at successful past products when developing their own projects - a modern, well-documented version could be quite valuable for them. As mentioned previously, Rogue is the origin of the roguelike genre, and thus holds historical and nostalgic value to its stakeholders. Additionally Prof. Smith, the 3XA3 TAs, and the project play-testers are stakeholders, as they will be evaluating the project and providing feedback. The rewrite should allow interested stakeholders to clearly examine the inner workings of the game, enable easy modification/extension of the game, and provide direct entertainment value like the original.



\end{document}
